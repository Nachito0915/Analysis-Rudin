\chapter{The Real and Complex Numbers}

\begin{enumerate}
\item
  If \(r\) is rational (\(r\neq 0\)) and \(x\) is irrational, prove that
  \(r + x\) and \(rx\) are irrational.
  \par\smallskip
  We will prove that \(r + x\) is irrational by
  \textit{reductio ad impossibilem}, contradiction.
  That is, \(p\to q\) becomes \(p\wedge\neg q\).
  Suppose \(r\) is rational (\(r\neq 0\)) and \(x\) is irrational and
  \(r + x\) is rational.
  Since \(r\) is rational, \(-r\) is rational and exist by the field axioms of
  addition.
  The sum of two rational numbers is rational by the closure property of
  \(\mathbb{Q}\).
  Then \(-r + (r + x) = (-r + r) + x = x\).
  We have reached a contradiction since \(x\) is clearly irrational.
  Therefore, \(r + x\) is irrational.
  \par\smallskip
  For the second statement, we will again use the argument of
  \textit{reductio ad impossibilem}.
  Since \(r\neq 0\) and rational, \(\frac{1}{r}\) is rational and exists by the
  field axioms of multiplication.
  The multiplication of two rational numbers is rational, again, by the
  closure property of \(\mathbb{Q}\).
  Then \(\frac{1}{r}(rx) = \bigl(\frac{1}{r}r\bigr)(x) = x\).
  We have reached a contradiction since \(x\) is irrational.
  That is, \(rx\) is irrational.
\item
  Prove that there is no rational number whose square is \(12\).
  \par\smallskip
  Suppose there is a rational number whose square is \(12\).
  Let \(\frac{a}{b}\) be this rational number.
  Then \(a^2 = 12b^2\).
  By the Fundamental Theorem of Arithmetic, we can write \(a\), \(b\), and
  \(12\) as a product of \textit{unique} primes.
  Let \(p_i\) and \(q_i\) be prime numbers and
  \(\alpha_i, \beta_i\in\mathbb{Z}^{\geq 0}\) for \(i = 1, 2, \ldots, n\).
  Then \(a = p_1^{\alpha_1}\cdot p_2^{\alpha_2}\cdots p_n^{\alpha_n}\),
  \(b = q_1^{\beta_1}\cdot q_2^{\beta_2}\cdots q_n^{\beta_n}\), and
  \(12 = 2^2\cdot 3\).
  We now have
  \begin{align*}
    (p_1^{\alpha_1}\cdot p_2^{\alpha_2}\cdots p_n^{\alpha_n})^2
    &= 2^2\cdot 3(q_1^{\beta_1}\cdot q_2^{\beta_2}\cdots q_n^{\beta_n})^2\\
    p_1^{2\alpha_1}\cdot p_2^{2\alpha_2}\cdots p_n^{2\alpha_n}
    &= 2^2\cdot 3(q_1^{2\beta_1}\cdot q_2^{2\beta_2}\cdots q_n^{2\beta_n})
      \eqnumtag\label{ch1prob2}
  \end{align*}
  Let \(p_k^{2\alpha_k}\) be \(3^{2\alpha_k}\) and \(q_m = 3^{2\beta_m}\).
  Then by \cref{ch1prob2}
  \begin{align*}
    3^{2\alpha_k} &= 3\cdot 3^{2\beta_m}\\
                  &= 3^{2\beta_m + 1}
  \end{align*}
  Therefore, \(2\alpha_k = 2\beta_m + 1\) which is a contradiction since an
  even number can never be an odd number.
  That is, there is no rational number whose square is \(12\).
\item
  Prove Proposition \(1.15\).
  \par\smallskip
  Proposition \(1.15\) states that the axioms for multiplication imply the
  following statements.
  \begin{enumerate}[label = (\alph*)]
  \item
    If \(x\neq 0\) and \(xy = xz\), then \(y = z\).
    \par\smallskip
    By the field axioms of multiplication, since \(x\neq 0\),
    \[
    y = 1\cdot y = \frac{1}{x}xy = \frac{1}{x}xz = \frac{1}{x}xz = z
    \]
    as was needed to be shown.
  \item
    If \(x\neq 0\) and \(xy = x\), then \(y = 1\).
    \par\smallskip
    Since \(x\neq 0\), we have
    \[
    y = 1\cdot y = \frac{1}{x}xy = \frac{1}{x}x = 1
    \]
    as was needed to shown.
  \item
    If \(x\neq 0\) and \(xy = 1\), then \(y = 1/x\).
    \par\smallskip
    Again, since we have that \(x\neq 0\),
    \[
    y = 1\cdot y = \frac{1}{x}xy = \frac{1}{x}\cdot 1 = \frac{1}{x}
    \]
    as was needed to be shown.
  \item
    If \(x\neq 0\), then \(1/(1/x) = x\)
    \par\smallskip
    Again, since we have that \(x\neq 0\),
    \[
    \frac{1}{1/x} = 1\cdot\frac{1}{1/x} = x\frac{1}{x}\frac{1}{1/x} =
    x\frac{1}{x}x = x
    \]
    as was needed to be shown.
  \end{enumerate}
\item
  Let \(E\) be a nonempty subset of an ordered set; suppose \(\alpha\) is a
  lower bound of \(E\) and \(\beta\) is an upper bound of \(E\).
  Prove that \(\alpha\leq\beta\).
  \par\smallskip
  Since \(E\neq\varnothing\), \(x\in E\).
  Since \(\alpha\) is a lower bound, \(\alpha\leq x\), and since \(\beta\) is
  an upper bound, \(\beta\geq x\).
  By the transitivity property, \(\alpha\leq\beta\).
\item
  Let \(A\) be a nonempty set of real numbers which is bounded below.
  Let \(-A\) be the set of all numbers \(-x\), where \(x\in A\).
  Prove that
  \[
  \inf A = -\sup(-A).
  \]
  Since \(A\) is nonempty and bounded below, \(A = \{x:x\in A\}\) and
  \(\inf(A) = \alpha\).
  Now, \(-A = \{-x:x\in A\}\) is also nonempty.
  Since \(\alpha\) is the infimum of \(A\), \(\alpha\leq x\) for all
  \(x\in A\).
  By multiplying by \(-1\), we get the following inequality
  \[
  \alpha\leq x\Rightarrow -\alpha\geq -x.
  \]
  That is, \(-\alpha\) is an upper bound of \(-A\).
  Suppose \(-\gamma = \sup(-A)\) and \(\varepsilon > 0\).
  Then \(-\gamma + \varepsilon\not\in -A\)
  \[
  -\alpha\geq -\gamma + \varepsilon\geq -\gamma\geq -x
  \]
  Again, by multiplying by negative one, we have
  \[
  \alpha\leq \gamma - \varepsilon\leq\gamma\leq x
  \]
  but \(\gamma - \varepsilon\notin A\) so \(\gamma\) is a lower bound of
  \(A\) which would contradict the fact that \(\alpha\) is the greatest lower
  bound of \(A\). 
  In order for \(\gamma\) to be the lower bound, \(\gamma = \alpha\) since the
  infimum is unique.
  So \(-\alpha = \sup(-A)\).
  Therefore, \(\alpha = \inf(A) = -\sup(-A) = -(-\alpha) = \alpha\).
\item
  Fix \(b > 1\).
  \begin{enumerate}[label = (\alph*)]
  \item
    If \(m\), \(n\), \(p\), \(q\) are integers, \(n, q > 0\), and
    \(r = m/n = p/q\), prove that
    \[
    (b^m)^{1/n} = (b^p)^{1/q}.
    \]
    Hence it makes sense to define \(b^r = (b^m)^{1/n}\).
    \par\smallskip
    Since \(n, q > 0\), \(nr = m = np/q\).
    \[
    (b^m)^{1/n} = (b^{np/q})^{1/n} = \bigl[(b^p)^{n/q}\bigr]^{1/n} =
    (b^p)^{1/q} = b^{p/q} = b^r
    \]
  \item
    Prove that \(b^{r + s} = b^rb^s\) if \(r\) and \(s\) are rational.
    \par\smallskip
    Let \(r = \frac{a}{b}\) and \(s = \frac{c}{d}\).
    Then
    \[
    b^{r + s} = b^{(ad + bc)/(bd)} = (b^{ad + bc})^{1/(bd)} =
    (b^a)^{1/b}(b^c)^{1/d} = b^rb^s
    \]
  \item
    If \(x\) is real, define \(B(x)\) to be the set of all numbers \(b^t\),
    where \(t\) is rational and \(t\leq x\).
    Prove that
    \[
    b^r = \sup B(r)
    \]
    when \(r\) is rational.
    Hence it makes sense to define
    \[
    b^x = \sup B(x)
    \]
    for every real \(x\).
    \par\smallskip
    From the statement \(b^r = \sup B(r)\), we see that \(b^r\in B(r)\).
    Let \(b^t\in B(r)\).
    Then \(b^r = b^tb^{r - t}\).
    Since \(b > 1\), \(b^t1^{r - t}\leq b^tb^{r - t} = b^r\); therefore,
    \(b^t\leq b^r\) for all \(b^t\in B(r)\) so \(b^r = \sup B(r)\).
  \item
    Prove that \(b^{x + y} = b^xb^y\) for all real \(x\) and \(y\).
    \par\smallskip
    
  \end{enumerate}
\item
  Fix \(b > 1\), \(y > 0\), and prove that there is a unique real \(x\) such
  that \(b^x = y\), by completing the following outline.
  (This is called the logarithm of \(y\) to the base of \(b\).)
  \begin{enumerate}[label = (\alph*)]
  \item
    For any positive integer \(n\), \(b^n - 1\geq n(b - 1)\).
    \par\smallskip
    From Theorem \(1.21\), we have that
    \(b^n - a^n = (b - a)(b^{n - 1} + b^{n - 2}a + \cdots a^{n - 1})\).
    Therefore, we now have
    \begin{align*}
      b^n - 1 & = (b - 1)(b^{n - 1} + b^{n - 2}1 + \cdots + ba^{n - 2} +
                1^{n - 1})\\
              & \geq (b - 1)(1^{n - 1} + 1^{n - 2}1 + \cdots + (1)1^{n - 2} +
                1^{n - 1})\eqnumtag\label{ch1prob7a}\\
              & = n(b - 1)1^{n - 1}\\
              & = n(b - 1)
    \end{align*}
    where \cref{ch1prob7a} occurs from letting \(b = 1\), and since \(b > 1\),
    we get the less than or equal to inequality.
  \item
    Hence \(b - 1\geq n(b^{1/n} - 1)\).
    \par\smallskip
    
  \item
    If \(t > 1\) and \(n > (b - 1)/(t - 1)\), then \(b^{1/n} < t\).
  \item
    If \(w\) is such that \(b^w < y\), then \(b^{w + 1/n} < y\) for
    sufficiently large \(n\); to see this, apply part \((c)\) with
    \(t = y\cdot b^{-w}\).
  \item
    If \(b^w > y\), then \(b^{w - 1/n} > y\) for sufficiently large \(n\).
  \item
    Let \(A\) be the set of all \(w\) such that \(b^w < y\), and show that
    \(x = \sup(A)\) satisfies \(b^x = y\).
  \item
    Prove that \(x\) is unique.
  \end{enumerate}
\item
  Prove that no order can be defined in the complex field that turns it into an
  ordered field.
  \textit{Hint: \(-1\) is a square}
  \par\smallskip
  Suppose that \(i > 0\).
  Then \(i^2 = -1 \ngtr 0\).
  Instead, let's suppose that \(i < 0\).
  Then \(i^4 = 1 \nless 0\).
  Therefore, \(\mathbb{C}\) is not ordered.
\item
  Suppose \(z = a + bi\), \(w = c + di\).
  Define \(z < w\) if \(a < c\), and also \(a = c\) but \(b < d\).
  Prove that this turns the set of all complex numbers into an ordered set.
  (This type of relation is called a \textit{dictionary order}, or
  \textit{lexicographic order}, for obvious reasons.)
  Does this ordered set have the least upper bound property?
  \par\smallskip
  The Law of Trichotomy states that a real number is either positive,
  negative, or zero.
  In otherwords, if \(x,y\in\mathbb{R}\), then \(x < y\), \(x = y\), or
  \(x > y\).
  Let \(a,b,c,d\in\mathbb{R}\).
  Then \(a < c\), \(a = c\), or \(a > c\).
  If \(a < c\), then \(z < w\).
  If \(a > c\), then \(z > w\).
  For \(a = c\), we have either \(b < d\), \(b = d\), or \(b > d\).
  If \(b < d\), then \(z < w\).
  If \(b > d\), them \(z > w\).
  Finally, if \(b = d\), then \(z = w\).
  Let \(z,w,u\in\mathbb{C}\) and \(a,b,c,d,e,f\in\mathbb{R}\) such that
  \(z\) and \(w\) are defined as above and \(u = e + if\).
  We need to show the tansitive property.
  That is, if \(z < w\) and \(w < u\), then \(z < u\).
  Since \(z < w\) and \(w < u\), we have that either \(a < c\) or \(a = c\) and
  \(b < d\) and \(c < e\) or \(c = e\) and \(d < f\).
  If \(a < c\) and \(c < e\), then \(a < e\) and \(z < u\).
  If \(a < c\), \(c = e\), and \(d < f\), then \(z < u\) since \(b < d < f\).
  If \(a = c\), \(b < d\), and \(c < e\), then \(a = c < e\) so \(z < u\).
  If \(a = c\), \(b < d\), \(c = e\), and \(d < f\), then \(a = c = e\) and
  \(b < d < f\) so \(z < u\).
  Thus, \(\mathbb{C}\) is an order set under the dictionary order.
  Since \(\mathbb{C}\) is an order set under the dictionary order, we have by
  the completeness axiom that \(\mathbb{C}\) with the dictionary order has the
  least upper bound property.
\item
  Suppose \(z = a + bi\), \(w = u + iv\), and
  \[
  a = \Bigl(\frac{\lvert w\rvert + u}{2}\Bigr)^{1/2},\qquad
  b = \Bigl(\frac{\lvert w\rvert - u}{2}\Bigr)^{1/2}.
  \]
  Prove that \(z^2 = w\) if \(v\geq 0\) and that \(\bar{z}^2 = w\) if
  \(v\leq 0\).
  Conclude that every complex number (with one exception!) has two complex
  square roots.
  \par\smallskip
  We have that \(z^2 = a^2 - b^2 + 2abi\) so \(a^2 - b^2 = u\).
  \begin{align*}
    2ab & = 2\Bigl(\frac{\lvert w\rvert + u}{2}
          \frac{\lvert w\rvert - u}{2}\Bigr)^{1/2}\\
        & = \pm\sqrt{\lvert w\rvert^2 - u^2}\\
        & = \pm v
  \end{align*}
  For \(v \geq 0\), \(z^2 = u + iv = w\).
  Now, \(\bar{z}^2 = a^2 - b^2 - 2abi\), so again we have \(a^2 - b^2 = u\) and
  \(-2ab = \mp v\).
  If \(v \leq 0\), then \(\bar{z}^2 = u + iv = w\).
  Therefore, all nonzero complex numbers have at least two complex square
  roots.
\item
  If \(z\) is a complex number, prove that there exists an \(r\geq 0\) and a
  complex number \(w\) with \(\lvert w\rvert = 1\) such that \(z = rw\).
  Are \(w\) and \(r\) always uniquely determined by \(z\)?
  \par\smallskip
  Since \(\lvert w\rvert = 1\), we can write \(w\) as
  \(w = \frac{z}{\lvert z\rvert}\).
  Then let \(r = \lvert z\rvert\) so \(z = rw\) where \(w\) and \(r\) are
  unique.
  If \(z = 0\), then \(r = 0\) and \(w\in\mathbb{C}\) such that
  \(\lvert w\rvert = 1\).
  Therefore, \(w\) is not unique.
\item
  If \(z_1,\ldots, z_n\) are complex, prove that
  \[
  \lvert z_1 + z_2 + \cdots + z_n\rvert\leq\lvert z_1\rvert + \lvert z_2\rvert
  + \cdots + \lvert z_n\rvert.
  \]
  First, we will show the triangle inequality is true for \(n = 2\) and use
  induction for \(n\geq 2\) and \(n\in\mathbb{Z}^+\).
  For \(n = 2\), we need to show
  \(\lvert z_1 + z_2\rvert\leq\lvert z_1\rvert + \lvert z_2\rvert\).
  \begin{align*}
    \lvert z_1 + z_2\rvert^2 & = (z_1 + z_2)(\bar{z}_1 + \bar{z}_2)\\
                             & = z_1\bar{z}_1 + z_1\bar{z}_2 + z_2\bar{z}_1 +
                               z_2\bar{z}_2\\
                             & \leq\lvert z_1\rvert^2 + 2\text{Re}(z_1z_2) +
                               \lvert z_2\rvert^2\\
                             & = \bigl(\lvert z_1\rvert + \lvert
                               z_2\rvert\bigr)^2 
  \end{align*}
  Taking square roots of the left and right sides, we have the desired results.
  Suppose this is true for \(k < n\).
  Then
  \[
  \lvert z_1 + \cdots + z_k\rvert\leq\lvert z_1\rvert + \cdots +
  \lvert z_k\rvert.
  \]
  Now, we need to show it is true for \(k + 1\).
  \begin{align*}
    \lvert z_1 + \cdots + z_{k + 1}\rvert
    & = \lvert (z_1 + \cdots + z_k) + z_{k + 1}\rvert\\
    & \leq\lvert z_1 + \cdots + z_k\rvert + \lvert z_{k + 1}\rvert\\
    & \leq\lvert z_1\rvert + \cdots + \lvert z_{k + 1}\rvert
  \end{align*}
  Therefore, by the principle of mathematical induction, the \(n\) dimensional
  triangle inequality is true.
\item
  If \(x, y\) are complex, prove that
  \[
  \bigl\lvert\lvert x\rvert - \lvert y\rvert\bigr\rvert\leq\lvert x - y\rvert.
  \]
  \par\smallskip
  Let \(x = x + y - y\).
  Then by the triangle inequality, we have
  \begin{align*}
    \lvert x + y - y\rvert
    & \leq \lvert x - y\rvert + \lvert y\rvert\\
    \lvert x\rvert & \leq \lvert x - y\rvert + \lvert y\rvert\\
    \lvert x\rvert - \lvert y\rvert & \leq \lvert x - y\rvert
  \end{align*}
  Similarly, we could let \(y = y + x - x\) and conclude
  \[
  \lvert y\rvert - \lvert x\rvert\leq\lvert x - y\rvert.
  \]
  Thus,
  \[
  \bigl\lvert\lvert x\rvert - \lvert y\rvert\bigr\rvert\leq\lvert x - y\rvert.
  \]
\item
  If \(z\) is a complex number such that \(\lvert z\rvert = 1\), that is, such
  that \(z\bar{z} = 1\), compute
  \[
  \lvert 1 + z\rvert^2 + \lvert 1 - z\rvert^2.
  \]
  We have that \(\lvert z\rvert^2 = z\bar{z}\) so
  \begin{align*}
    \lvert 1 + z\rvert^2 + \lvert 1 - z\rvert^2
    & = (1 + z)(1 - \bar{z}) + (1 - z)(1 - \bar{z})\\
    & = 2 + z + \bar{z} + 2 - z - \bar{z}\\
    & = 4
  \end{align*}
\item
  Under what conditions does equality hold in the Schwarz inequality?
  \par\smallskip
  The Schwarz inequality (also known as the Cauchy-Schwarz inequality) is
  \[
  \Bigl\lvert\sum_j^n a_j\bar{b}_j\Bigr\rvert^2\leq\sum_j^n\lvert a_j\rvert^2
  \sum_j^n\lvert b_j\rvert^2.
  \]
  Let \(A = \sum\lvert a_j\rvert^2\), \(B = \sum\lvert \bar{b}_j\rvert^2\), and
  \(C = \sum\lvert a_j\bar{b}_j\rvert^2\).
  From the proof in the book, we ahve \(0 = B(AB - \lvert C\rvert^2)\).
  Therefore, equality holds if \(B = 0\) or \(AB - \lvert C\rvert^2 = 0\).
\item
  Suppose \(k\geq 3\), \(\mathbf{x},\mathbf{y}\in\mathbb{R}^k\),
  \(\lvert\mathbf{x} - \mathbf{y}\rvert = d > 0\), and \(r > 0\).
  Prove:
  \begin{enumerate}[label = (\alph*)]
  \item
    If \(2r > d\), there are infinitely many \(\mathbf{z}\in\mathbb{R}^k\) such
    that
    \[
    \lvert\mathbf{z} - \mathbf{x}\rvert = \lvert\mathbf{z} - \mathbf{y}\rvert
    = r.
    \]
  \item
    If \(2r = d\), there exactly one such \(\mathbf{z}\).
  \item
    If \(2r < d\), there is no such \(\mathbf{z}\)
  \end{enumerate}
  How must these statements be modified if \(k\) is \(2\) or \(1\)?
\item
  Prove that
  \[
  \lvert\mathbf{x} + \mathbf{y}\rvert^2 + \lvert\mathbf{x} - \mathbf{y}\rvert^2
  = 2\lvert\mathbf{x}\rvert^2 + 2\lvert\mathbf{y}\rvert^2
  \]
  if \(\mathbf{x}\in\mathbb{R}^k\) and \(\mathbf{y}\in\mathbb{R}^k\).
  Interpret this geometrically, as a statement about parallelograms.
  \par\smallskip
  We have that
  \begin{align*}
    \lvert\mathbf{x} + \mathbf{y}\rvert^2
    & = (\mathbf{x} + \mathbf{y}) (\mathbf{x} + \mathbf{y})\\
    & = \lvert\mathbf{x}\rvert^2 + 2\mathbf{x}\cdot\mathbf{y} +
      \lvert\mathbf{y}\rvert^2\eqnumtag\label{ch1prob17a}\\
    \lvert\mathbf{x} - \mathbf{y}\rvert^2
    & = (\mathbf{x} - \mathbf{y}) (\mathbf{x} - \mathbf{y})\\
    & = \lvert\mathbf{x}\rvert^2 - 2\mathbf{x}\cdot\mathbf{y} +
      \lvert\mathbf{y}\rvert^2\eqnumtag\label{ch1prob17b}\\
    \intertext{Then by adding \cref{ch1prob17a,ch1prob17b}, we have}
    \lvert\mathbf{x} + \mathbf{y}\rvert^2 +
    \lvert\mathbf{x} - \mathbf{y}\rvert^2
    & = 2\lvert\mathbf{x}\rvert^2 + 2\lvert\mathbf{y}\rvert^2
  \end{align*}
  Then \(\mathbf{x} + \mathbf{y}\) is the longer diagonal of the parallelogram
  and \(\mathbf{x} - \mathbf{y}\) is the shorter diagonal of the parallelogram
  see \cref{ch1prob17}.
  \begin{figure}[H]
    \centering
    \includestandalone[mode = image, width = 2in]{Tikz/ch1prob17}
    \caption{The parallelogram for vectors \(\mathbf{x}\) and \(\mathbf{y}\).}
    \label{ch1prob17}
  \end{figure}
  Then the sum of squares of the diagonals of a parallelogram are equal to the
  sum of the squares of the sides of the parallelogram.
\item
  If \(k\geq 2\) and \(\mathbf{x}\in\mathbb{R}^k\), prove that there exists
  \(\mathbf{y}\in\mathbb{R}^k\) such that \(\mathbf{y}\neq\mathbf{0}\) but
  \(\mathbf{x}\cdot\mathbf{y} = 0\).
  Is this also true if \(k = 1\)?
  \par\smallskip
  If \(\mathbf{x} = 0\), then the components of \(\mathbf{y}\) can be any real
  numbers.
  If \(\mathbf{x}\neq 0\), then
  \(\mathbf{y} =
  \begin{bmatrix} -x_k & -x_{k - 1} & \cdots & -x_1\end{bmatrix}^{\intercal}\).
  For \(k = 1\), this is not true since for the multiplication of any two
  nonzero real numbers is nonzero.
\item
  Suppose \(\mathbf{a}\in\mathbb{R}^k\), \(\mathbf{b}\in\mathbb{R}^k\).
  Find \(\mathbf{c}\in\mathbb{R}^k\) and \(r > 0\) such that
  \[
  \lvert\mathbf{x} - \mathbf{a}\rvert = 2\lvert\mathbf{x} - \mathbf{b}\rvert
  \]
  if and only if \(\lvert\mathbf{x} - \mathbf{c}\rvert = r\).
  (Solutions \(3\mathbf{c} = 4\mathbf{b} - \mathbf{a}\),
  \(3r = 2\lvert\mathbf{b} - \mathbf{a}\rvert\).)
\end{enumerate}
%%% Local Variables:
%%% mode: latex
%%% TeX-master: t
%%% End:
