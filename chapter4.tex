\chapter{Continuity}
\label{ch4}

\begin{exercise}
\item
  Suppose \(f\) is a real function defined on \(\mathbb{R}^1\) which satisfies
  \[
  \lim_{h\to 0}\bigl[f(x + h) - f(x - h)\bigr] = 0
  \]
  for every \(x\in\mathbb{R}^1\).
  Does this imply \(f\) is continuous?
  \par\smallskip
  Consider
  \[
  f(x) =
  \begin{cases}
    1, & x = 0\\
    0, & x\neq 0
  \end{cases}
  \]
  Then \(f\) is continuous at zero if for every \(\epsilon > 0\) there exist a
  \(\delta > 0\) such that \(d_Y\bigl[f(x),f(0)\bigr] < \epsilon\) for all
  \(x\in\mathbb{R}^1\) for which \(0 < d_X(x,0) < \delta\).
  For \(x\neq 0\), \(0 < \lvert x - 0\rvert < \delta\).
  Then \(\lvert f(x) - f(0)\rvert = 1\).
  Take \(\epsilon = 1/2\) so \(\lvert f(x) - f(0)\rvert > \epsilon\) and \(f\)
  is not continuous at \(x = 0\).
  For \(x\neq 0\), \(f(x) = 0\).
  Now for \(h < \lvert x\rvert\), \(x + h\neq 0\neq x - h\); therefore,
  \(f(x + h) = f(x - h) = 0\).
  \[
  \lim_{h\to 0}\bigl[f(x + h) - f(x - h)\bigr] = 0
  \]
  For \(x = 0\) and \(h > 0\), \(f(x + h) = f(h) = f(-h) = f(x - h)\) and again
  the limit is zero.
\item
  If \(f\) is a continuous mapping of a metric space \(X\) into a metric space
  \(Y\), prove that
  \[
  f(\bar{E})\subset\overline{f(E)}
  \]
  for every set \(E\subset X\).
  Show, by example, that \(f(\bar{E})\) can be a proper subset of
  \(\overline{f(E)}\).
  \par\smallskip
  Let \(E\subset X\) and \(x\in\bar{E}\).
  Let \(V\) be an open neighborhood of \(f(x)\).
  Since \(f\) is continuous, \(f^{-1}(V)\) is an open set of \(X\) where
  \(x\in f^{-1}(V)\).
  Therefore, \(f^{-1}(V)\) intersects \(E\).
  Let \(y\in f^{-1}(V)\cap E\).
  Then
  \begin{align*}
    f[f^{-1}(v)\cap E] & = f[f^{-1}(V)]\cap f(E)\\
                       & = V\cap f(E)
  \end{align*}
  so \(f(y)\in V\cap f(E)\).
  Therefore, \(f(y)\) exist in a neighborhood of \(f(x)\) so \(f(x)\) is an
  adherent point so \(f(x)\in\overline{f(E)}\).
  Thus, \(f(\bar{E})\subset\overline{f(E)}\).
\item
  \label{4.3}
  Let \(f\) be a continuous real function on a metric space \(X\).
  Let \(Z(f)\) (the zero set of \(f\)) be the set of all \(p\in X\) at which
  \(f(p) = 0\).
  Prove that \(Z(f)\) is closed.
  \par\smallskip
  Let \(E\) be the set that contains \(f(p) = 0\).
  The only element in \(E\) is \(0\) so \(E\) is closed.
  Since \(f\) is continuous, \(f\) and \(f^{-1}\) map closed sets to closed
  sets.
  Thus, \(f^{-1}(E)\subset X\) and is closed in \(X\).
  Now, \(f^{-1}(E)\) are all the points \(p\in X\) such that \(f(p) = 0\) so
  \(f^{-1}(E) = Z(f)\) and \(Z(f)\) is closed.
\item
  \label{4.4}
  Let \(f\) and \(g\) be continuous mappings of a metric space \(X\) into a
  metric space \(Y\), and let \(E\) be a dense subset of \(X\).
  Prove that \(f(E)\) is dense in \(f(X)\).
  If \(g(p) = f(p)\) for all \(p\in E\), prove that \(g(p) = f(p)\) for all
  \(p\in X\).
  (In other words, a continuous mapping is determined by its values on a dense
  subset of its domain.)
  \par\smallskip
  Let \(y\in f(X)\) and set \(y = f(x)\) for \(x\in X\).
  Since \(f\) in continuous, for any \(\epsilon > 0\), there exists
  \(\delta > 0\) such that whenever \(d_X(x,z) < \delta\),
  \(d_Y[f(x),f(z)] < \epsilon\).
  Since \(E\) is dense in \(X\), \(z\in E\) for this choice of \(\delta\).
  Now \(w = f(z)\in f(E)\).
  Then \(d_Y[f(x),f(z)] = d_Y(y,w) < \epsilon\).
  Therefore, we have that \(f(E)\) is dense in \(f(X)\) since for any
  \(y\in f(X)\), and all \(\epsilon > 0\), we have \(w\in f(E)\) such that
  \(d_Y(y,w) < \epsilon\).
  Let \(x\in X\) and \(n = 1/\epsilon\) where \(\epsilon > 0\).
  Let \(\{p_n\}\) be a sequence in \(E\).
  For \(n > N\), we have \(\epsilon = 1/n\).
  Then \(d(x,p_n) < 1/n = \epsilon\); therefore, \(\lim_{n\to\infty}p_n = x\).
  Since \(f\) and \(g\) are continuous, we have
  \[
  f(x) = f\Bigl(\lim_{n\to\infty}p_n\Bigr) = \lim_{n\to\infty}f(p_n)
  = \lim_{n\to\infty}g(p_n) = g\Bigl(\lim_{n\to\infty}p_n\Bigr) = g(x)
  \]
  as was needed to be shown.
\item
  \label{4.5}
  If \(f\) is a real continuous function defined on a closed set
  \(E\subset\mathbb{R}^1\), prove that there exist continuous real functions
  \(g\) on \(\mathbb{R}^1\) such that \(g(x) = f(x)\) for all \(x\in E\).
  (Such functions \(g\) are called \textit{continuous extensions} of \(f\) from
  \(E\) to \(\mathbb{R}^1\).)
  Show that the result becomes false if the word "closed" is omitted.
  Extend the result to vector valued functions.
  \textit{Hint: let the graph of \(g\) be a straight line on each of the
    segments which constitute the complement of \(E\) (compare
    \cref{2.29}~\cref{ch2}).
    The result remains true if \(\mathbb{R}^1\) is replaced by any metric
    space, but the proof is not so simple.}
\item
  If \(f\) is defined on \(E\), the \textit{graph} of \(f\) is the set of
  points \((x,f(x))\), for \(x\in E\).
  In particular, if \(E\) is a set of real numbers, and \(f\) is real-valued,
  the graph of \(f\) is a subset of the plane.
  Suppose \(E\) is compact, and prove that \(f\) is continuous on \(E\) if and
  only if its graph is compact.
\item
  If \(E\subset X\) and if \(f\) is a function defined on \(X\), the
  \textit{restriction} of \(f\) to \(E\) is the function \(g\) whose domain of
  definition is \(E\), such that \(g(p) = f(p)\) for \(p\in E\).
  Define \(f\) and \(g\) on \(\mathbb{R}^2\) by: \(f(0,0) = g(0,0) = 0\),
  \(f(x,y) = xy^2/(x^2 + y^4)\), \(g(x,y) = xy^2/(x^2 + y^6)\) if
  \((x,y)\neq 0\).
  Prove that \(f\) is bounded on \(\mathbb{R}^2\), that \(g\) is unbounded in
  every neighborhood of \((0,0)\), and that \(f\) is not continuous at
  \((0,0)\); nevertheless, the restrictions of both \(f\) and \(g\) to every
  straight line in \(\mathbb{R}^2\) are continuous!
  \par\smallskip
  If \(f\) is bounded, there exists \(M > 0\) such that
  \[
  \Bigl\lvert\frac{xy^2}{x^2 + y^4}\Bigr\rvert\leq M.
  \]
  Now, \(x^2,y^4,y^2\geq 0\) for all \(x,y\in\mathbb{R}\) so we have
  \begin{align*}
    \frac{\lvert x\rvert y^2}{x^2 + y^4} & \leq M\\
    \lvert x\rvert y^2 & \leq M(x^2 + y^4)\\
    0 & \leq x^2 - \lvert x\rvert y^2/M + y^4
  \end{align*}
  For \(M = 1/2\), we have \((\lvert x\rvert - y^2)^2\geq 0\) and
  \[
  \Bigl\lvert\frac{xy^2}{x^2 + y^4}\Bigr\rvert\leq\frac{1}{2}.
  \]
  Thus, \(f\) is bounded.
  Let \(x_n\) be the sequence \(\{1/n^3\}\) and \(y_n\) be the sequence
  \(\{1/n\}\).
  Then \((x_n,y_n)\to (0,0)\) as \(n\to\infty\).
  \[
  \lim_{n\to\infty}g(x_n,y_n) = \lim_{n\to\infty}\frac{x_ny_n^2}{x_n^2 + y_n^6}
  = \lim_{n\to\infty}\frac{1/n^5}{2/n^6} = \lim_{n\to\infty}\frac{n}{2} =
  \infty
  \]
  Therefore, \(g\) is not bounded in a neighborhood of \((0,0)\).
  For \(f\) to be continuous at \((0,0)\), \(\lim_{x\to p}f(x) = f(p)\) if and
  only if \(f(p_n)\to f(p)\) for all sequences \(p_n\to p\).
  Let \(x_n\) and \(y_n\) be the following sequences \(\{1/n^2\}\) and
  \(\{1/n\}\), respectively.
  Then \((x_n,y_n)\to (0,0)\) as \(n\to\infty\).
  \[
  \lim_{n\to\infty}f(x_n, y_n) = \lim_{n\to\infty}\frac{1/n^4}{2/n^4} =
  \frac{1}{2}\neq 0
  \]
  Thus, \(f\) is not continuous at \((0,0)\).
  If we restrict both \(f\) and \(g\) to straight lines, we must show \(f\) and
  \(g\) are continuous at \((0,0)\).
  The only straight lines that go through \((0,0)\) are \(y = x = 0\) and
  \(y = mx\) for \(m\in\mathbb{R}\).
  \begin{align*}
    \lim_{(x, y)\to (0, y)}f(x, y) & = \frac{0\cdot y^2}{0 + y^4}\\
                                   & = 0\\
    \lim_{(x, y)\to (x, 0)}f(x, y) & = \frac{x\cdot 0}{x^2 + 0}\\
                                   & = 0\\
    \lim_{(x, y)\to (0, y)}g(x, y) & = \frac{0\cdot y^2}{0 + y^6}\\
                                   & = 0\\
    \lim_{(x, y)\to (x, 0)}g(x, y) & = \frac{x\cdot 0}{x^2 + 0}\\
                                   & = 0\\
    \lim_{(x, y)\to (x, mx)}f(x, y) & = \frac{m^2x^3}{x^2 + m^4x^4}\\
                                   & = \frac{m^2x}{1 + m^4x^2}\eqnumtag
                                     \label{4.7a}\\
    \lim_{(x, y)\to (x, mx)}g(x, y) & = \frac{m^2x^3}{x^2 + m^6x^6}\\
                                   & = \frac{m^2x}{1 + m^6x^4}\eqnumtag
                                     \label{4.7b}\\
  \end{align*}
  Then as \(x\to 0\), \cref{4.7a,4.7b} go to zero; thus, when \(f\) and \(g\)
  are restricted to straight lines, \(f\) and \(g\) are continuous at
  \((0,0)\).
\item
  Let \(f\) be a real uniformly continuous function on the bounded set \(E\) in
  \(\mathbb{R}^1\).
  Prove that \(f\) is bounded on \(E\).
  Show that the conclusion is false if boundedness of \(E\) is omitted from the
  hypothesis.
  \par\smallskip
  Since \(E\) is a bounded set in \(\mathbb{R}\), there exist
  \(M\in\mathbb{R}\) such that for all \(x\in E\),
  \(\lvert x\rvert\leq M < \infty\).
  Therefore, \(E\subset [-M,M]\).
  Since \(f\) is uniformly continuous, for all \(\epsilon > 0\), there exist a
  \(\delta > 0\) such that \(\lvert x - y\rvert < \delta\) whenever
  \(\lvert f(x) - f(y)\rvert < \epsilon\) for \(x,y\in E\).
  Let \(n\) be an integer such that \(1/n < \delta\).
  Let's construct a cover, \(S\), of \([-M,M]\).
  \[
  S = \bigl\{[-M, -M + 1/n], [-M + 1/n, -M + 2/n],\ldots,[M - 2/n, M - 1/n],
  [M - 1/n, M]\bigr\}
  \]
  Let \(E\subset\bigcup_{k = 1}^NI_k\) such that \(I_k\cap E\neq\varnothing\)
  where \(I_k\) is an interval in \(S\).
  For each \(k\in[1,N]\), choose \(x_k\in I_k\cap E\).
  Then for \(x_k,x\in I_k\cap E\), we have
  \(\lvert x_k - x\rvert > 1/n < \delta\).
  Therefore,
  \begin{align*}
    \lvert f(x_k) - f(x)\rvert
    & \leq -\lvert f(x_k)\rvert + \lvert f(x)\rvert\\
    \lvert f(x)\rvert - \lvert f(x_k)\rvert & < \epsilon\\
    \lvert f(x)\rvert & < \epsilon + \lvert f(x_k)\rvert
  \end{align*}
  Let \(S = \max_{1\leq k\leq N}\{1 + f(x_k)\}\).
  Then for all \(x\in E\), \(\lvert f(x)\rvert\leq S\) so \(f\) is bounded on
  \(E\).
  \par\smallskip
  Let \(E = \mathbb{R}\).
  Since \(f(x)\) is uniformly continuous, for all \(\epsilon > 0\), there exist
  a \(\delta > 0\) such that \(\lvert x - y\rvert < \delta\) whenever
  \(\lvert f(x) - f(y)\rvert < \epsilon\).
  Let \(\delta = \epsilon\) and \(f(x) = x\).
  Then
  \[
  \lvert f(x) - f(y)\rvert = \lvert x - y\rvert < \delta = \epsilon
  \]
  but \(f(x) = x\) is not bounded on \(\mathbb{R} = E\).
\item
  \label{4.9}
  Show that the requirement in the definition of uniform continuity can be
  rephrased as follows, in terms of diameters of sets: To every
  \(\epsilon > 0\) there exists a \(\delta > 0\) such that
  \(\diam f(E) < \epsilon\) for all \(E\subset X\) with \(\diam E < \delta\).
\item
  Complete the details of the following alternate  proof of theorem \(4.19\):
  If \(f\) is not uniformly continuous, then for some \(\epsilon > 0\) there
  are sequences \(\{p_n\},\{q_n\}\in X\) such that \(d_X(p_n,q_n)\to 0\) but
  \(d_Y(f(p_n),f(q_n)) > \epsilon\).
  Use theorem \(2.37\) to obtain a contradiction.
\item
  Suppose \(f\) is a uniformaly continuous mapping of a metric space \(X\)
  into a metric space \(Y\) and prove that \(\{f(x_n)\}\) is a Cauchy sequence
  in \(Y\) for every Cauchy sequence \(\{x_n\}\) in \(X\).
  Use this result to give an alternative proof of the theorem stated in
  \cref{4.13}
\item
  A uniformly continuous function of a uniformly continuous function is
  uniformly continuous.
  State this more precisely and prove it.
\item
  \label{4.13}
  Let \(E\) be a dense subset of a metric space \(X\), and let \(f\) be a
  uniformly continuous \textit{real} function defined on \(E\).
  Prove that \(f\) has a continuous extension from \(E\) to \(X\)
  (see \cref{4.5} for terminology).
  (Uniqueness follows from \cref{4.4})
  \textit{Hint: For each \(p\in X\) and each positive integer \(n\), let
    \(V_n(p)\) be the set of all \(q\in E\) with \(d(p,q) < 1/n\).}
  Use \cref{4.9} to show that the intersection of the closures of the sets
  \(f(V_1(p)),f(V_2(p)),\ldots\), consists of a single point, say \(g(p)\), of
  \(\mathbb{R}^1\).
  Prove that the function \(g\) so defined on \(X\) is the desired extension
  of \(f\).
  Could the range space \(\mathbb{R}^1\) be replaced by \(\mathbb{R}^k\)?
  By any compact metric space?
  By any complete metric space?
  By any metric space?
\item
  Let \(I = [0,1]\) be the close unit interval.
  Suppose \(f\) is a continuous mapping of \(I\) into \(I\).
  Prove that \(f(x) = x\) for at least one \(x\in I\).
\item
  Call a mapping of \(X\) into \(Y\) \textit{open} if \(f(V)\) is an open set
  in \(Y\) whenever \(V\) is an open set in \(X\).
  Prove that every continuous open mapping of \(\mathbb{R}^1\) into
  \(\mathbb{R}^1\) is monotonic.
\item
  Let \([x]\) denote the largest integer contained in \(x\), that is, \([x]\)
  is the integer such that \(x - 1 < [x]\leq x\); and let \((x) = x - [x]\)
  denote the fractional part of \(x\).
  What discontinuities do the functions \([x]\) and \((x)\) have?
\item
  Let \(f\) be a real function defined on \((a,b)\).
  Prove that the set of points at which \(f\) has a simple discontinuity is at
  most countable.
  \textit{Hint: Let \(E\) be the set on which \(f(x-) < f(x+)\).
    With each point \(x\) of \(E\), associate a triple \((p,q,r)\) of rational
    numbers such that}
  \begin{exercise}[label = (\alph*)]
  \item
    \(f(x-) < p < f(x+)\),
  \item
    \(a < q < t < x\) implies \(f(t) < p\),
  \item
    \(x < t < r < b\) implies \(f(t) > p\).
  \end{exercise}
  The set of all such triples is countable.
  Show that each triple is associated with at most one point of \(E\).
  Deal similarly with the other possible types of simple discontinuities.
\item
  Every rational \(x\) can be written in the form \(x = m/n\), where \(n > 0\),
  and \(m\) and \(n\) are integers without any common divisor.
  When \(x = 0\), we take \(n = 1\).
  Consider the function \(f\) defined on \(\mathbb{R}^1\) by
  \[
  f(x) =
  \begin{cases}
    0, & x\text{ irrational}\\
    \frac{1}{n}, & x = \frac{m}{n}
  \end{cases}
  \]
  Prove that \(f\) is continuous at every irrational point, and that \(f\) has
  a simple discontinuity at every rational point.
\item
  Suppose \(f\) is a real function with domain \(\mathbb{R}^1\) which has the
  intermediate value property: If \(f(a) < c < f(b)\), then \(f(x) = c\) for
  some \(x\) between \(a\) and \(b\).
  Suppose also, for every rational \(r\), that the set of all \(x\) with
  \(f(x) = r\) is closed.
  Prove that \(f\) is continuous.
  \textit{Hint: If \(x_n\to x_0\) but \(f(x_n) > r > f(x_0)\) for some \(r\)
    and all \(n\), then \(f(t_n) = r\) for some \(t_n\) between \(x_0\) and
    \(x_n\); thus, \(t_n\to x_0\).
    Find a contradiction.}
  (N. J. Fine, \textit{Amer. Math. Monthly}, vol. 73, 1966, p. 782.)
\item
  If \(E\) is a nonempty subset of a metric space \(X\), define the distance
  from \(x\in X\) to \(E\) by
  \[
  \rho_E(x) = \inf_{z\in E} d(x, z).
  \]
  \begin{exercise}[label = (\alph*)]
  \item
    Prove that \(\rho_E(x) = 0\) if and only if \(x\in\bar{E}\).
  \item
    Prove that \(\rho_E\) is a uniformly continuous function on \(X\), by
    showing that
    \[
    \lvert\rho_E(x) - \rho_E(y)\rvert\leq d(x, y)
    \]
    for all \(x\in X\), \(y\in Y\).
    \textit{Hint: \(\rho_E(x)\leq d(x,z)\leq d(x,y) + d(y,z)\) so that
      \(\rho_E(x)\leq d(x,y) + \rho_E(y)\).}
  \end{exercise}
\item
  Suppose \(K\) and \(F\) are disjoint sets in a metric space \(X,K\) is
  compact, \(F\) is closed.
  Prove that there exists \(\delta > 0\) such that \(d(p,q) > \delta\) if
  \(p\in K\), \(q\in F\).
  \textit{Hint: \(\rho_F\) is a continuous positive function on \(K\).}
  Show that the continuous may fail for two disjoint closed sets if neither is
  compact.
\item
  Let \(A\) and \(B\) be disjoint nonempty closed sets in a metric space \(X\),
  and define
  \[
  f(p) = \frac{\rho_A(p)}{\rho_A(p) + \rho_E(p)}
  \]
  for \(p\in X\).
  Show that \(f\) is a continuous function on \(X\) whose range lies in
  \([0,1]\), that \(f(p) = 0\)  precisely on \(A\) and \(f(p) = 1\) precisely
  on \(B\).
  This establishes a converse of \cref{4.3}: Every closed set \(A\subset X\) is
  \(Z(f)\) for some continuous real \(f\) on \(X\).
  Setting
  \[
  V = f^{-1}([0, 1/2])\qquad W = f^{-1}([1/2, 1])
  \]
  show that \(V\) and \(W\) are open and disjoint, and that \(A\subset V\),
  \(B\subset W\).
  (Thus pairs of disjoint closed sets in a metric space can be covered by pairs
  of disjoint open sets.
  This property of metric spaces is called \textit{normality}.)
\item
  A real-valued function \(f\) defined in \((a,b)\) is said to be
  \textit{convex} if
  \[
  f(\lambda x + (1 - \lambda)y)\leq\lambda f(x) + (1 - \lambda)f(y)
  \]
  whenever \(a < x < b\), \(a < y < b\), \(0 < \lambda < 1\).
  Prove that every convex function is continuous.
  (For example, if \(f\) is convex, so is \(e^f\).)
  If \(f\) is convex in \((a,b)\) and if \(a < s < t < u < b\), show that
  \[
  \frac{f(t) - f(s)}{t - s}\leq\frac{f(u) - f(s)}{u - s}\leq
  \frac{f(u) - f(t)}{u - t}.
  \]
\item
  Assume that \(f\) is a continuous real function defined in \((a,b)\) such
  that
  \[
  f\Bigl(\frac{x + y}{2}\Bigr)\leq \frac{f(x) + f(y)}{2}
  \]
  for all \(x,y\in (a,b)\).
  Prove that \(f\) is convex.
\item
  If \(A\subset\mathbb{R}^k\) and \(B\subset\mathbb{R}^k\), define \(A + B\) to
  be the set of all sums \(\mathbold{x + y}\) with \(\mathbold{x}\in A\),
  \(\mathbold{y}\in B\).
  \begin{exercise}[label = (\alph*)]
  \item
    If \(K\) is compact
  \end{exercise}
\end{exercise}

%%% Local Variables:
%%% mode: latex
%%% TeX-master: t
%%% End:
