\chapter{Numerical Sequences and Series}
\label{ch3}

\begin{exercise}
\item
  Prove that convergence of \(\{s_n\}\) implies convergence of
  \(\{\lvert s_n\rvert\}\).
  Is the converse true?
  \par\smallskip
  Since \(\{s_n\}\) converges, it is Cauchy.
  Let \(\epsilon > 0\) be given.
  There exist \(n,m > N\) such that \(\lvert s_n - s_m\rvert < \epsilon\) since
  \(\{s_n\}\) is Cauchy.
  \begin{align*}
    \lvert s_n\rvert & = \lvert s_n - s_m + s_m\rvert\\
                     & \leq \lvert s_n - s_m\rvert + \lvert s_m\rvert\\
    \lvert s_n\rvert - \lvert s_m\rvert & \leq \lvert s_n - s_m\rvert\\
    \intertext{Similarly, we can show}
    \lvert s_m\rvert - \lvert s_n\rvert & \leq \lvert s_m - s_n\rvert\\
                     & = \lvert s_n - s_m\rvert
  \end{align*}
  so
  \[
  \bigl\lvert\lvert s_n\rvert - \lvert s_m\rvert\bigr\rvert\leq
  \lvert s_n - s_m\rvert < \epsilon.
  \]
  No.
  Consider the sequence \(\{s_n\} = (-1)^n\).
  Let \(\epsilon = 1\).
  If \(\{s_n\}\) converges, it will converge to \(\pm 1\).
  WLOG assume \(s_n\to 1\).
  Let \(n > N\) such that \(n\) is odd.
  \[
  \lvert (-1)^n - 1\rvert = \lvert -1 - 1\rvert = 2\not < \epsilon
  \]
  Therefore, the sequence \(\{s_n\}\) doesn't converge.
  However, \(\{\lvert s_n\rvert\}\) does converge to \(1\).
  Let \(\epsilon > 0\) given.
  There exist an \(n > N\) such that
  \(\bigl\lvert\lvert (-1)^n\rvert - 1\bigr\rvert < \epsilon\).
  For any \(n > N\), \((-1)^n = \pm 1\) and \(\lvert \pm 1\rvert = 1\).
  \[
  \bigl\lvert\lvert (-1)^n\rvert - 1\bigr\rvert =
  \lvert 1 - 1\rvert = 0 < \epsilon
  \]
\item
  Caclulate \(\lim_{n\to\infty}\sqrt{n^2 + n} - n\).
  \begin{align*}
    \lim_{n\to\infty}\sqrt{n^2 + n} - n
    & = \lim_{n\to\infty}\Bigl(\sqrt{n^2 + n} - n\Bigr)
      \frac{\sqrt{n^2 + n} + n}{\sqrt{n^2 + n} + n}\\
    & = \lim_{n\to\infty}\frac{n}{\sqrt{n^2 + n} + n}\\
    & = \lim_{n\to\infty}\frac{n}{\sqrt{n^2 + n} + n}\frac{1/n}{1/n}\\
    & = \lim_{n\to\infty}\frac{1}{\sqrt{1 + 1/n} + 1}\\
    & = \frac{1}{2}
  \end{align*}
\item
  If \(s_1 = \sqrt{2}\) and
  \[
  s_{n + 1} = \sqrt{2 + \sqrt{s_n}}
  \]
  \(n\in\mathbb{Z}^+\) prove that \(\{s_n\}\) converges, and that \(s_n < 2\)
  for \(n\in\mathbb{Z}^+\).
  \par\smallskip
  Let \(s_{n + 1} = \sqrt{2 + \sqrt{s_n}}\) be written as
  \[
  s = \sqrt{2 + s}\Rightarrow s^2 - s - 2 = 0.
  \]
  Then \(\sqrt{2 + \sqrt{s_n}} < \sqrt{2 + s}\).
  Since we are dealing with real numbers, we are only looking for positive
  \(s\).
  \[
  s^2 - s - 2 = (s - 2)(s + 1) = 0\eqnumtag\label{3.2}
  \]
  so \(s = 2, -1\).
  Thus, \(s_n < 2\) so \(\{s_n\}\) is bounded above by two.
  Additionally, since \(s_1 = \sqrt{2}\), we have that
  \(\sqrt{2}\leq s_n < 2\).
  The parabola is concave up and symmetrical about \(s = 1/2\).
  That is, \(s\) monotonically increases from \((1/2, \infty)\) so \(\{s_n\}\)
  monotonoically increases on \(\bigl[\sqrt{2}, 2\bigr)\).
  Theorem \(3.14\) states that monotonic sequences converge if and only if it
  is bounded.
  Therefore, since \(\{s_n\}\) is bounded and monotonic, \(\{s_n\}\) converges
  and it converges to \(2\).
\item
  Find the upper and lower limits of the sequence \(\{s_n\}\) defined by
  \[
  s_1 = 0,\qquad s_{2m} = \frac{s_{2m - 1}}{2},\qquad
  s_{2m + 1} = \frac{1}{2} + s_{2m}.
  \]
  Let's determine a few of the terms.
  Then \(s_{2m} = \{0,1/4,3/8,7/16,\ldots\}\) and
  \(s_{2m + 1} = \{1/2,3/4,7/8,15/16,\ldots\}\) or we can write them as
  \begin{align*}
    s_{2m} & = \frac{1}{2} - \frac{1}{2^m}\\
    s_{2m + 1} & = 1 - \frac{1}{2^m}
  \end{align*}
  The \(\lim_{n\to\infty}\sup s_n = \lim_{n\to\infty} 1 - \frac{1}{2^n} = 1\)
  and
  \(\lim_{n\to\infty}\inf s_n = \lim_{n\to\infty} \frac{1}{2} - \frac{1}{2^n}
  = \frac{1}{2}\).
\item
  For any two real sequences \(\{a_n\}\), \(\{b_n\}\), prove that
  \[
  \lim_{n\to\infty}\sup (a_n + b_n)\leq\lim_{n\to\infty}\sup a_n +
  \lim_{n\to\infty}\sup b_n,
  \]
  provided the sum on the right is not of the form \(\infty - \infty\).
  \par\smallskip
  Let \(\lim_{n\to\infty}\sup a_n = a\), \(\lim_{n\to\infty}\sup b_n = b\), and
  \(\lim_{n\to\infty}\sup (a_n + b_n) = c\) where \(a_n + b_n = c_n\).
  Let \(\epsilon > 0\) be given.
  Then there exist \(N_1,N_2\in\mathbb{Z}^+\) such that for \(n\geq N_1\) and
  \(n\geq N_2\)
  \begin{align*}
    \lvert a_n - a\rvert & < \frac{\epsilon}{2}\\
    \lvert b_n - b\rvert & < \frac{\epsilon}{2}
  \end{align*}
  Let \(N = \max\{N_1,N_2\}\).
  Then when \(n\geq N\),
  \begin{align*}
    \lvert c_n - c\rvert & = \lvert a_n + b_n - (a + b)\rvert
                           \eqnumtag\label{3.5.a}\\
                         & \leq \lvert a_n - a\rvert + \lvert b_n - b\rvert
                           \eqnumtag\label{3.5.b}\\
                         & < \frac{\epsilon}{2} + \frac{\epsilon}{2}\\
                         & = \epsilon
  \end{align*}
  From \cref{3.5.a,3.5.b}, we have that
  \[
  \lim\sup c_n\leq\lim\sup a_n + \lim\sup b_n,
  \]
  and since \(c_n = a_n + b_n\), the identity follows.
\item
  Investigate the behavior (convergence or divergence) of \(\sum a_n\) if
  \begin{exercise}[label = (\alph*)]
  \item
    \(a_n = \sqrt{n + 1} - \sqrt{n}\)
    \par\smallskip
    Let \(S_N\) be the \(N\)th partial sum.
    Then
    \[
    S_N = \sum_{n = 0}^N\bigl(\sqrt{n + 1} - \sqrt{n}\bigr) =
    1 + \sqrt{2} - 1 + \sqrt{3} - \sqrt{2} + \cdots + \sqrt{N} - \sqrt{N - 1}
    + \sqrt{N + 1} - \sqrt{N}
    \]
    Therefore, the \(S_N = \sqrt{N + 1}\).
    \[
    \lim_{N\to\infty}S_N = \lim_{N\to\infty}\sqrt{N + 1} = \infty
    \]
    so the series doesn't converge.
  \item
    \(a_n = \frac{\sqrt{n + 1} - \sqrt{n}}{n}\)
    \par\smallskip
    Let's re-write the series by multiplying by the conjugate.
    \[
    \sum_{n = 1}^{\infty}\frac{\sqrt{n + 1} - \sqrt{n}}{n}
    \frac{\sqrt{n + 1} + \sqrt{n}}{\sqrt{n + 1} + \sqrt{n}} =
    \sum_{n = 1}^{\infty}\frac{1}{n\bigl(\sqrt{n + 1} + \sqrt{n}\bigr)}
    \]
    Now
    \[
    \sum_{n = 1}^{\infty}\frac{1}{n\bigl(\sqrt{n + 1} + \sqrt{n}\bigr)}\leq
    \sum_{n = 1}^{\infty}\frac{1}{2n\sqrt{n}}
    \]
    By theorem \(3.28\), \(\sum\frac{1}{n^p}\) converges if \(p > 1\) and
    diverges if \(p\leq 1\).
    Therefore,
    \[
    \sum_{n = 1}^{\infty}\frac{1}{2n\sqrt{n}} < \infty
    \]
    since \(p = 3/2\) so
    \[
    \sum_{n = 1}^{\infty}\frac{1}{n\bigl(\sqrt{n + 1} + \sqrt{n}\bigr)} <
    \infty.
    \]
  \item
    \(a_n = \bigl(\sqrt[n]{n} - 1\bigr)^n\)
    \par\smallskip
    By the root test,
    \[
    \lim_{n\to\infty}\sqrt[n]{\lvert\sqrt[n]{n} - 1\rvert^n} =
    \lim_{n\to\infty}\lvert\sqrt[n]{n} - 1\rvert.
    \]
    Let \(x_n = \sqrt[n]{n} - 1\).
    Then
    \[
    n = (x_n - 1)^n = \sum_{k = 0}^n\binom{n}{k}x_n^k = 1 + nx_n +
    \frac{n(n-1)}{2}x_n^2 + \cdots
    \]
    so \(\frac{n(n-1)}{2}x_n^2 < n\Rightarrow x_n^2 < \frac{2}{n - 1}
    \Rightarrow x_n < \sqrt{\frac{2}{n - 1}}\).
    \begin{align*}
      \lim_{n\to\infty}\lvert\sqrt[n]{n} - 1\rvert
      & = \lim_{n\to\infty}\lvert x_n\rvert\\
      & < \lim_{n\to\infty}\sqrt{\frac{2}{n - 1}}\\
      & = 0
    \end{align*}
    Since \(\sqrt{\frac{2}{n - 1}}\) converges and
    \(x_n < \sqrt{\frac{2}{n - 1}}\), \(x_n\) converges.
    By the root and comparison test,
    \[
    \sum_{n = 1}^{\infty}\bigl(\sqrt[n]{n} - 1\bigr)^n
    \]
    converges.
  \item
    \(a_n = \frac{1}{1 + z^n}\) for complex values of \(z\).
    \[
    \sum_{n = 0}^{\infty}\frac{1}{1 + z^n}\leq
    \sum_{n = 0}^{\infty}\frac{1}{z^n}
    \]
    and \(\sum_{n = 0}^{\infty}\frac{1}{z^n}\) converges for
    \(\bigl\lvert\frac{1}{z}\bigr\rvert < 1\).
  \end{exercise}
\item
  Prove that the convergence of \(\sum a_n\) implies the convergence of
  \[
  \sum\frac{\sqrt{a_n}}{n},
  \]
  if \(a_n\geq 0\).
  \par\smallskip
  For \(a_n\geq 1\), \(\sqrt{a_n}\leq a_n\).
  By the comparison test,
  \[
  \sum\frac{\sqrt{a_n}}{n} < \sum a_n < \infty
  \]
  so for \(a_n\geq 1\), the series converges.
  For \(0\leq a_n < 1\), \(a_n\leq\sqrt{a_n}\).
  We can write all rationals and irrational numbers in \([0,1)\) as \(b/n\)
  for \(b\in\mathbb{R}^{\geq 0}\).
  Then
  \[
  \sum\frac{\sqrt{a_n}}{n} = \frac{\sqrt{b}}{n\sqrt{n}} < \infty
  \]
  since \(p = 3/2 > 1\) so the series converges.
\item
  If \(\sum a_n\) converges, and if \(\{b_n\}\) is monotonic and bounded,
  prove that \(\sum a_nb_n\) converges.
  \par\smallskip
  By theorem \(3.14\), we know that monotonic bounded sequences converge.
  Let \(\{b_n\}\to M\) for some number \(M < \infty\) or
  \(\lvert b_n\rvert\leq M\).
  Since \(\sum a_n\) converges, for a given \(\epsilon > 0\) and \(k\geq N\),
  \(m\geq k\geq N\) implies that
  \[
  \Bigl\lvert\sum_{n = k}^ma_n\Bigr\rvert\leq\sum_{n = k}^m\lvert a_n\rvert\leq
  \epsilon.
  \]
  Take \(\epsilon = \frac{\epsilon}{M}\).
  Then
  \[
  \Bigl\lvert\sum a_nb_n\Bigr\rvert\leq\sum\lvert a_n\rvert\lvert b_n\rvert\leq
  \sum\lvert a_n\rvert M.
  \]
  Since \(\sum\lvert a_n\rvert\leq\epsilon/M\), the result follows; that is,
  \[
  \sum\lvert a_n\rvert M\leq\epsilon
  \]
  so \(\sum a_nb_n\) converges.
\item
  Find the radius of convergence of each of the following power series:
  \begin{exercise}[label = (\alph*)]
  \item
    \(\sum n^3z^n\)
    \par\smallskip
    Here will use the ratio test.
    \[
    \limsup_{n\to\infty}\Bigl\lvert\frac{(n + 1)^3z^{n + 1}}{n^3z^n}\Bigr\rvert
    = \lvert z\rvert\limsup_{n\to\infty}\Bigl\lvert\frac{n + 1}{n}\Bigr\rvert^3
    = \lvert z\rvert\limsup_{n\to\infty}\Bigl\lvert\frac{n}{n}\Bigr\rvert^3 =
    \lvert z\rvert
    \]
    Then \(\limsup = \alpha\) and \(R = \frac{1}{\alpha}\) so \(R = 1\) and
    \(\lvert z\rvert < 1\) for convergence.
  \item
    \(\sum\frac{2^n}{n!}z^n\)
    \par\smallskip
    Again, we use the ratio test.
    \[
    \limsup_{n\to\infty}\Bigl\lvert
    \frac{2^{n + 1}z^{n + 1}n!}{2^nz^n(n + 1)!}\Bigr\rvert =
    2\lvert z\rvert\limsup_{n\to\infty}\Bigl\lvert\frac{1}{n + 1}\Bigr\rvert =
    0
    \]
    Thus, \(R = \infty\).
  \item
    \(\sum\frac{2^n}{n^2}z^n\)
    \par\smallskip
    Following the same test, we have
    \[
    \limsup_{n\to\infty}\Bigl\lvert
    \frac{2^{n + 1}z^{n + 1}n^2}{2^nz^n(n + 1)^2}\Bigr\rvert =
    2\lvert z\rvert\limsup_{n\to\infty}\Bigl\lvert\frac{n}{n}\Bigr\rvert^2 =
    2\lvert z\rvert
    \]
    so \(R = 1/2\) and \(\lvert z\rvert < 1/2\) for convergence.
  \item
    \(\sum\frac{n^3}{3^n}z^n\)
    \[
    \limsup_{n\to\infty}\Bigl\lvert
    \frac{(n + 1)^3z^{n + 1}3^n}{3^{n + 1}z^nn^3}\Bigr\rvert =
    \frac{\lvert z\rvert}{3}
    \limsup_{n\to\infty}\Bigl\lvert\frac{n}{n}\Bigr\rvert^3 =
    \frac{\lvert z\rvert}{3}
    \]
    so \(R = 3\) and \(\lvert z\rvert < 3\) for convergence.
  \end{exercise}
\item
  Suppose the the coefficients of the power series \(\sum a_nz^n\) are
  integers, infinitely many of which are distinct from zero.
  Prove that the radius of convergence is at most \(1\).
  \par\smallskip
\item
  Suppose \(a_n > 0\), \(s_n = a_1 + \cdots + a_n\), and \(\sum a_n\) diverges.
  \begin{exercise}[label = (\alph*), ref = \arabic{exercisei} (\alph*)]
  \item
    Prove that \(\sum\frac{a_n}{1 + a_n}\) diverges.
    \par\smallskip
    Theorem \(3.23\) states that if \(\sum a_n\) converges, then
    \(\lim_{n\to\infty}a_n = 0\).
    Since \(\sum a_n\) doesn't converge, there exist no \(M\) such that
    \(a_n\leq M\) for \(M\in\mathbb{R}\).
    Then
    \[
    \lim_{n\to\infty}\frac{a_n}{1 + a_n}\geq
    \lim_{n\to\infty}\frac{M}{1 + M} = 1
    \]
    for some \(M\gg 10^8\).
    Therefore, the limit is greater than or equal to one so
    \(\sum\frac{a_n}{1 + a_n}\) doesn't converge.
  \item
    \label{3.11.b}
    Prove that
    \[
    \frac{a_{N + 1}}{s_{N + 1}} + \cdots + \frac{a_{N + k}}{s_{N + k}}\geq 1
    - \frac{s_N}{s_{N + k}}
    \]
    and deduce that \(\sum\frac{a_n}{s_n}\) diverges.
    \par\smallskip
    Each partial sum \(s_N\) increase so
    \[
    \frac{a_{N + 1}}{s_{N + 1}} + \cdots + \frac{a_{N + k}}{s_{N + k}}\geq
    \frac{1}{s_{N + k}}(a_{N + 1} + \cdots + a_{N + k})  =
    1 - \frac{s_N}{s_{N + k}}
    \]
    since \(a_{N + 1} + \cdots + a_{N + k} = s_{N + k} - s_N\).
    Now,
    \[
    \sum\frac{a_n}{s_n} = \frac{a_1}{a_1} + \frac{a_2}{a_1 + a_2} + \cdots
    + \frac{a_n}{\sum a_n} + \cdots
    \]
    Let \(\epsilon > 0\) be given.
    Then
    \[
    \Bigl\lvert 1 - \frac{s_N}{s_{N + k}}\Bigr\rvert > \epsilon
    \]
    since for \(k\) sufficiently large, \(s_{N + k}\to\infty\).
    That is, \(\lvert 1 - s_N/s_{N + k}\rvert\) can be made larger than
    \(1/2\).
    Take \(\epsilon = 0.1\) and the series falls to converge.
  \item
    Prove that
    \[
    \frac{a_n}{s_n^2}\leq\frac{1}{s_{n - 1}} - \frac{1}{s_n}
    \]
    and deduce that \(\sum\frac{a_n}{s_n^2}\) converges.
    \par\smallskip
    We can write \(\frac{1}{s_{n - 1}} - \frac{1}{s_n}\) as
    \[
    \frac{1}{s_{n - 1}} - \frac{1}{s_n} = \frac{s_n - s_{n - 1}}{s_ns_{n - 1}}
    \]
    where \(s_n = a_n + \sum_{i = 1}^{n - 1}a_i\) and
    \(s_{n - 1} = \sum_{i = 1}^{n - 1}a_i\) so \(s_n - s_{n - 1} = a_n\).
    Now \(s_n^2\geq s_ns_{n - 1}\) so
    \(\frac{1}{s_n^2}\leq\frac{1}{s_ns_{n - 1}}\).
    \[
    \frac{1}{s_{n - 1}} - \frac{1}{s_n} = \frac{a_n}{s_ns_{n - 1}}\geq
    \frac{a_n}{s_n^2}
    \]
    The telescoping series
    \[
    \sum_{n = 2}^{\infty}\frac{1}{s_{n - 1}} - \frac{1}{s_n}\geq
    \sum_{n = 1}^{\infty}\frac{a_n}{s_n^2}.
    \]
    Since
    \[
    \sum_{n = 2}^N\frac{1}{s_{n - 1}} - \frac{1}{s_n} =
    \frac{1}{s_1} - \frac{1}{s_2} + \frac{1}{s_2} - \frac{1}{s_3} + \cdots +
    \frac{1}{s_{N - 1}} - \frac{1}{s_N} + \frac{1}{s_N} - \frac{1}{s_{N + 1}}
    = \frac{1}{s_1} - \frac{1}{s_{N + 1}}
    \]
    and \(\sum a_n\to\infty\), \(\lim_{N\to\infty}\frac{-1}{s_{N + 1}} =
    \lim_{N\to\infty}\frac{-1}{a_{N + 1}} = \frac{-1}{\infty} = 0\) so
    \[
    \sum_{n = 2}^{\infty}\frac{1}{s_{n - 1}} - \frac{1}{s_n} = \frac{1}{a_1}
    \]
    and \(\sum\frac{a_n}{s_n^2} < \infty\).
  \item
    What can be said about
    \[
    \sum\frac{a_n}{1 + na_n}\quad\text{and}\quad\sum\frac{a_n}{1 + n^2a_n}
    \mbox{?}
    \]
    \par\smallskip
    For the second series, we have
    \[
    \sum\frac{a_n}{1 + n^2a_n}\leq\sum\frac{1}{n^2} < \infty.
    \]
    For the first series, suppose \(a_n\in\mathbb{R}\), then
    \[
    \sum\frac{a_n}{1 + na_n}\leq\sum\frac{1}{n}\to\infty
    \]
    Suppose \(a_n = 1/n^{1 + p}\).
    Then
    \[
    \sum\frac{a_n}{1 + na_n} =
    \sum\frac{1/n^{1 + p}}{1 + n\bigl(1/n^{1 + p}\bigr)} =
    \sum\frac{1}{n^{1 + p} + n}\leq\sum\frac{1}{n^{1 + p}} < \infty
    \]
    for \(p > 0\).
    Otherwise, the series diverges to infinity.
  \end{exercise}
\item
  Suppose \(a_n > 0\) and \(\sum a_n\) converges.
  Put
  \[
  r_n = \sum_{m = n}^{\infty}a_m.
  \]
  \begin{exercise}[label = (\alph*)]
  \item
    Prove that
    \[
    \frac{a_m}{r_m} + \cdots + \frac{a_n}{r_n} > 1 - \frac{r_n}{r_m}
    \]
    if \(m < n\), and deduce that \(\sum\frac{a_n}{r_n}\) diverges.
    \[
    \frac{a_m}{r_m} + \cdots + \frac{a_n}{r_n} >
    \frac{a_m + \cdots + a_n}{r_m} = \frac{r_m - r_n}{r_m} =
    1 - \frac{r_n}{r_m}
    \]
    By the same reasoning as \cref{3.11.b}, the series doesn't converge.
  \item
    Prove that
    \[
    \frac{a_n}{\sqrt{r_n}} < 2\bigl(\sqrt{r_n} - \sqrt{r_{n + 1}}\bigr)
    \]
    and deduce that \(\sum\frac{a_n}{\sqrt{r_n}}\) converges.
    \par\smallskip
    Consider \(\sqrt{r_n} - \sqrt{r_{n + 1}}\).
    \begin{align*}
      \sqrt{r_n} - \sqrt{r_{n + 1}}
      & = \sqrt{r_n} - \sqrt{r_{n + 1}}
        \frac{\sqrt{r_n} + \sqrt{r_{n + 1}}}{\sqrt{r_n} + \sqrt{r_{n + 1}}}\\
      & = \frac{r_n - r_{n + 1}}{\sqrt{r_n} + \sqrt{r_{n + 1}}}\\
      & = \frac{a_n}{\sqrt{r_n} + \sqrt{r_{n + 1}}}\\
      & > \frac{a_n}{2\sqrt{r_n}}\\
      2\bigl(\sqrt{r_n} - \sqrt{r_{n + 1}}\bigr) & > \frac{a_n}{\sqrt{r_n}}
    \end{align*}
    Let's consider the series of
    \begin{align*}
      2\sum_{n = 1}^{\infty}\bigl(\sqrt{r_n} - \sqrt{r_{n + 1}}\bigr)
      & = 2\lim_{N\to\infty}\sum_{n = 1}^N
        \bigl(\sqrt{r_n} - \sqrt{r_{n + 1}}\bigr)\\
      & = 2\lim_{N\to\infty}\bigl[\sqrt{r_1} - \sqrt{r_2} + \sqrt{r_2} -
        \sqrt{r_3} + \cdots + \sqrt{r_N} - \sqrt{r_{N + 1}}\bigr]\\
      & = 2\lim_{N\to\infty}\bigl(\sqrt{r_1} - \sqrt{r_{N + 1}}\bigr)\\
      & = 2\sqrt{r_1} - \lim_{N\to\infty}
        \biggl(\sum_{m = N + 1}^{\infty}a_m\biggr)^{1/2}\\
        % & = 2\sqrt{r_1} - 
        % \biggl(\lim_{N\to\infty}\sum_{m = N + 1}^{\infty}a_m\biggr)^{1/2}\\
      \intertext{Since \(\sum a_n\) converges, \(\sum_{m = N + 1}^{\infty}a_n\)
      can be made less than \(\epsilon > 0\).}
      & = 2\sqrt{r_1}
    \end{align*}
    By the comparison test, \(\sum\frac{a_n}{\sqrt{r_n}}\) converges.
  \end{exercise}
\item
  Prove that the Cauchy product of two absolutely convergent series converges
  absolutely.
  \par\smallskip
  Let \(\sum a_n\) and \(\sum b_n\) be two absolutely convergent series.
  Then \(\sum\lvert a_n\rvert < M\) and \(\sum\lvert b_n\rvert < n\).
  Let \(c_n = \sum_k^ma_kb_{n - k}\).
  \begin{align*}
    \sum_{n = 0}^m\lvert c_n\rvert
    & = \sum_{n = 0}^m\Bigl\lvert\sum_{k = 0}^n a_kb_{n - k}\Bigr\rvert\\
    & \leq \sum_{n = 0}^m\sum_{k = 0}^n\lvert a_kb_{n - k}\rvert\\
    & = \lvert a_0b_0\rvert + \lvert a_0b_1\rvert + \lvert a_1b_0\rvert +
      \cdots + \lvert a_0b_m\rvert + \lvert a_1b_{m - 1}\rvert + \cdots +
      \lvert a_{m - 1}b_1 + a_mb_0\rvert\\
    & = \lvert a_0\rvert\lvert b_0\rvert + \lvert a_0\rvert\lvert b_1\rvert +
      \lvert a_1\rvert\lvert b_0\rvert + \cdots + \lvert a_0\rvert
      \lvert b_m\rvert + \lvert a_1\rvert\lvert b_{m - 1}\rvert + \cdots +
      \lvert a_{m - 1}\rvert\lvert b_1\rvert + \lvert a_m\rvert
      \lvert b_0\rvert\\
    & = \sum_{n = 0}^m\lvert a_n\rvert\sum_{k = 0}^{m - n}\lvert b_k\rvert\\
    & < M\sum_{k = 0}^{m - n}\lvert b_k\rvert\\
    & < MN
  \end{align*}
  Therefore, the Cauchy product of two absolutely convergent series converge.
\item
  If \(\{s_n\}\) is a complex sequence, define its arithmetic means
  \(\sigma_n\) by
  \[
  \sigma_n = \frac{s_0 + \cdots + s_n}{n + 1}
  \]
  for \(n = 0,1,\ldots\).
  \begin{exercise}[label = (\alph*), ref = \arabic{exercisei} (\alph*)]
  \item
    \label{3.12.a}
    If \(\lim s_n = s\), prove that \(\lim\sigma_n = s\).
    \par\smallskip
    Since \(\{s_n\}\to s\), for \(\epsilon > 0\), there exists \(n > N\) such
    that \(\lvert s_n - s\rvert < \epsilon/2\).
    Let \(N_0 = \max\Bigl\{N,\frac{4(N + 1)\lvert s\rvert}{\epsilon}\Bigr\}\).
    For \(n > N_0\),
    \begin{align*}
      \lvert\sigma_n - s\rvert
      & = \Bigl\lvert\frac{s_0 + \cdots + s_n}{n + 1} - s\Bigr\rvert\\
      & = \Bigl\lvert\frac{s_0 - s + \cdots + s_n - s}{n + 1}\Bigr\rvert\\
      & \leq \Bigl\lvert\frac{s_0 - s + \cdots + s_N - s}{n + 1}\Bigr\rvert +
        \Bigl\lvert\frac{s_{N + 1} - s + \cdots + s_n - s}{n + 1}\Bigr\rvert\\
      & < \frac{m_1}{n + 1}\lvert s_N - s\rvert + \frac{m_2}{n + 1}
        \lvert s_n - s\rvert\\
      \intertext{where \(m_1,m_s < n + 1\)}
      & < \lvert s_N - s\rvert + \lvert s_n - s\rvert\\
      & < \frac{\epsilon}{2} + \frac{\epsilon}{2}\\
      & = \epsilon
    \end{align*}
    Thus, \(\lim\sigma_n = s\).
  \item
    Construct a sequence \(\{s_n\}\) which does not converge, although
    \(\lim\sigma_n = 0\).
    \par\smallskip
    Let \(\{s_n\} = (-1)^n\).
    Then
    \[
    \sigma_n = \frac{1 - 1 + 1 - \cdots + 1}{n + 1} =
    \begin{cases}
      0, & \text{if \(n\) is odd}\\
      \frac{1}{n + 1}, & \text{if \(n\) is even}
    \end{cases}
    \]
    Now taking the limit of \(\sigma_n\), we have that \(\lim\sigma_n = 0\).
  \item
    Can it happen that \(s_n > 0\) for all \(n\) and that \(\limsup = \infty\),
    although \(\lim\sigma_n = 0\).
    \par\smallskip
    Yes.
    Let \(\{s_n\} = \log(\log(n + 1))\) for \(n\geq 2\).
    Then
    \[
    \limsup_{n\to\infty} s_n = \infty.
    \]
    We can write \(\sigma_n\) as
    \[
    \sigma_n = \frac{\log(\log(3)) + \log(\log(4)) + \cdots +
      \log(\log(n + 1))}{n + 1}\leq\frac{\log(n\log(n))}{n + 1}
    \]
    Now taking the limit of \(\sigma_n\), we have
    \begin{align*}
      \lim_{n\to\infty}\sigma_n
      & = \lim_{n\to\infty}\frac{s_n}{n + 1}\\
      & \leq \lim_{n\to\infty}\frac{\log(n\log(n))}{n + 1}\\
      & = \lim_{n\to\infty}\frac{\log(n)}{n + 1} +
        \lim_{n\to\infty}\frac{\log(\log(n))}{n + 1}\\
      & = \lim_{n\to\infty}\frac{\log(\log(n))}{n + 1}\\
      & = 0
    \end{align*}
  \item
    Put \(a_n = s_n - s_{n - 1}\), for \(n\geq 1\).
    Show that
    \[
    s_n - \sigma_n = \frac{1}{n + 1}\sum_{k = 1}^nka_k.
    \]
    Assume that \(\lim na_n = 0\) and that \(\{\sigma_n\}\) converges.
    Prove that \(\{s_n\}\) converges.
    [This gives converse of \cref{3.12.a}, but under the additional assumption
    that \(na_n\to 0\).]
    \par\smallskip
    Recall that \(s_n = \sum_{k = 0}^na_k\).
    Then the left hand side can be written as
    \begin{align*}
      s_n - \sigma_n & = a_0 + \cdots + a_n -
                       \frac{s_0 + \cdots + s_n}{n + 1}\\
                     & = a_0 + \cdots + a_n -
                       \frac{(n + 1)a_0 + na_1 + \cdots + a_n}{n + 1}\\
                     & = \frac{a_1 + 2a_2 + \cdots + na_n}{n + 1}\\
                     & = \frac{1}{n + 1}\sum_{k = 0}^nka_k
    \end{align*}
    as was needed to be shown.
  \item
    Derive the last conclusion from a weaker hypothesis: Assume \(M < \infty\),
    \(\lvert na_n\rvert\leq M\) for all \(n\), and \(\lim\sigma_n = \sigma\).
    Prove that \(\lim s_n = \sigma\), by completing the following outline:
    \par\smallskip
    If \(m < n\), then
    \par\smallskip
    {\color{NavyBlue}
      we have that
      \begin{align*}
        \sigma_n - \sigma_m
        & = \frac{s_0 + \cdots + s_n}{n + 1} -
          \frac{s_0 + \cdots + s_m}{m + 1}\\
        & = (s_0 + \cdots + s_n)\frac{m - n}{(n + 1)(m + 1)} +
          \frac{1}{m + 1}\sum_{i = m + 1}^ns_i\\
        & = \frac{m - n}{m + 1}\sigma_n + \frac{1}{m + 1}
          \sum_{i = m + 1}^ns_i\\
        \intertext{Let's multiple through by \(\frac{m + 1}{m - n}\).}
        (\sigma_n - \sigma_m)\frac{m + 1}{m - n}
        & = \sigma_n - \frac{1}{n - m}\sum_{i = m + 1}^ns_i\\
        -\sigma_n & = (\sigma_n - \sigma_m)\frac{m + 1}{n - m} -
                    \frac{1}{n - m}\sum_{i = m + 1}^ns_i
      \end{align*}
      Finally, we just need to add \(s_n\) to both sides.
      Note that \(\sum_{i = m + 1}^n1 = 1\) so take
      \(s_n = \sum_{i = m + 1}^ns_n\).
      Then we obtain the desired result.
    }
    \[
    s_n - \sigma_n = \frac{m + 1}{n - m}(\sigma_n - \sigma_m) +
    \frac{1}{n - m}\sum_{i = m + 1}^n(s_n - s_i).
    \]
    For these \(i\),
    \par\smallskip
    {\color{NavyBlue}
      we have
      \begin{align*}
        \lvert s_n - s_i\rvert
        & = \lvert a_n + a_{n - 1} + \cdots + a_{i + 1}\rvert\\
        & \leq \lvert a_n\rvert + \cdots + \lvert a_{i + 1}\rvert\\
        \intertext{By the hypothesis, \(\lvert na_n\rvert\leq M\) so
        \(\lvert a_n\rvert\leq M/n\).}
        & \leq \frac{M}{n} + \cdots + \frac{M}{i + 1}\\
        & = M\Bigl(\frac{1}{n} + \cdots + \frac{1}{i + 1}\Bigr)\\
        \intertext{Now, \(i + 1\) is the smallest indices so
        \(\frac{1}{i + 1}\) is the largest fraction and we have \(n - i\)
        fractions.}
        & \leq \frac{M(n - i)}{i + 1}
      \end{align*}
      Plugging in \(i = m + 1\), we achieve the desired results.
    }
    \[
    \lvert s_n - s_i\rvert\leq\frac{(n - i)M}{i + 1}\leq
    \frac{(n - m - 1)M}{m + 2}.
    \]
    Fix \(\epsilon > 0\) and associate with each \(n\) the integer \(m\) that
    satifies
    \[
    m\leq\frac{n - \epsilon}{1 + \epsilon} < m + 1
    \]
    Then \((m + 1)/(n - m)\leq 1/\epsilon\) and
    \(\lvert s_n - s_i\rvert < M\epsilon\).
    Hence
    \[
    \limsup_{n\to\infty}\lvert s_n - \sigma\rvert\leq M\epsilon.
    \]
    Since \(\epsilon\) was arbitrary, \(\lim s_n = \sigma\).
  \end{exercise}
\item
  Definition \(3.21\) can be extended to the case in which the \(a_n\) lie in
  some fixed \(\mathbb{R}^k\).
  Absolute convergence is defined as convergence of \(\sum\lvert a_n\rvert\).
  Show that Theorems \(3.22,3.23,3.25(a),3.33,3.34,3.42,3.45,3.47\), and
  \(3.55\) are true in this more general setting.
  (Only slight modifications are required in any of the proofs.)
  \par\smallskip
  \textbf{Theorem} \(\mathbold{3.22}\): \(\sum\mathbold{a}_n\) converges if
  and only if for every \(\epsilon > 0\) there is an integer \(N\) such that
  \[
  \Bigl\lvert\sum_{k = n}^m\mathbold{a}_k\Bigr\rvert\leq\epsilon
  \]
  if \(m\geq n\geq N\).
  \par\smallskip
  For \(\lvert a_i - b_i\rvert\leq\lvert\mathbold{a} - \mathbold{b}\rvert\leq
  \sum_{i = 1}^k\lvert a_i - b_i\rvert\), the sequence \(\{\mathbold{a}_n\}\)
  converges if and only if each subsequence \(\{a_{n_j}\}\) converges for
  \(j = 1,\ldots,k\).
  That is, the sequences converge if they are Cauchy; therefore, the vector
  sequence is Cauchy.
  \par\smallskip
  \textbf{Theorem} \(\mathbold{3.23}\): If \(\sum\mathbold{a}_n\) converges,
  then \(\lim_{n\to\infty}\mathbold{a}_n = \mathbold{0}\).
  \par\smallskip
  From theorem \(3.22\), we have that \(\sum\mathbold{a}_n\) converges if each
  \(\{a_{n_j}\}\) converges for \(j = 1,\ldots,k\).
  Thus, \(a_{n_j}\to 0\) for each \(j\) so \(\mathbold{a}_n\to 0\) or
  \(\lim_{n\to\infty}\mathbold{a}_n = \mathbold{0}\).
  \par\smallskip
  \textbf{Theorem} \(\mathbold{3.25(a)}\): If
  \(\lvert\mathbold{a}_n\rvert\leq c_n\) for \(n\geq N_0\), where \(N_0\) is
  some fixed integer, and if \(\sum c_n\) converges, then
  \(\sum\mathbold{a}_n\) converges.
  \par\smallskip
  By the hypothesis, \(a_{n_j}\) converges for each \(j\), and since each
  subsequences converges, \(\sum\mathbold{a}_n\) converges.
  \par\smallskip
  \textbf{Theorem} \(\mathbold{3.33}\): Given \(\sum\mathbold{a}_n\), put
  \(\alpha = \limsup_{n\to\infty}\sqrt[n]{\lvert\mathbold{a}_n\vert}\).
  Then
  \begin{exercise}[label = (\alph*)]
  \item
    if \(\alpha < 1\), \(\sum\mathbold{a}_n\) converges;
    \par\smallskip
    From the previous theorems, we have that
    \(\sqrt[n]{\lvert a_{n_j}\rvert}\leq\sqrt[n]{\lvert\mathbold{a}_n\rvert}\).
    Now, if \(\alpha < 1\), then each subsequence converges; therefore,
    \(\sum\mathbold{a}_n\) converges.
  \item
    if \(\alpha > 1\), \(\sum\mathbold{a}_n\) diverges; and
    \par\smallskip
    When \(\alpha > 1\), \(\lvert\mathbold{a}_n\rvert > 1\) for infinitely
    many \(n\).
    Therefore, the series diverges.
  \item
    if \(\alpha = 1\), the test gives no information.
  \end{exercise}
  \textbf{Theorem} \(\mathbold{3.34}\): The series \(\sum\mathbold{a}_n\)
  \begin{exercise}[label = (\alph*)]
  \item
    converges if \(\limsup_{n\to\infty}\frac{\lvert\mathbold{a}_{n + 1}\rvert}
    {\lvert\mathbold{a}_n\rvert} < 1\),
    \par\smallskip
    The limes superior inequality means that for some \(\epsilon > 0\) and
    constant \(M\), \(\lvert\mathbold{a}_n\rvert M\epsilon ^n\).
    Thus, \(\sum\mathbold{a}_n\) converges absolutely so series converges by
    theorem \(3.25\).
  \item
    diverges if \(\frac{\lvert\mathbold{a}_{n + 1}\rvert}
    {\lvert\mathbold{a}_n\rvert}\geq 1\) for \(n\leq n_0\), whenever \(n_0\) is
    some fixed integer.
    \par\smallskip
    From the inequality, we get that \(\mathbold{a}_n\) doesn't go to zero.
    Therefore, the series doesn't converge.
  \end{exercise}
  \textbf{Theorem} \(\mathbold{3.42}\): Suppose
  \begin{exercise}[label = (\alph*)]
  \item
    the partial sum \(\mathbold{A}_n\) of \(\sum\mathbold{a}_n\) form a bounded
    sequence;
  \item
    \(b_0\geq b_1\geq b_2\geq\cdots\);
  \item
    \(\lim_{n\to\infty}b_n = 0\)
  \end{exercise}
  Then \(\sum b_n\mathbold{a}_n\) converges.
  \par\smallskip
  Choose \(M\) such that \(\lvert\mathbold{A}_n\rvert\leq M\) for all \(n\).
  Given \(\epsilon > 0\), there is an integer \(N\) such that
  \(b_N\leq\frac{\epsilon}{2M}\).
  For \(N\leq p\leq q\), we have
  \begin{align*}
    \Bigl\lvert\sum_{n = p}^q\mathbold{a}_nb_n\Bigr\rvert
    & = \Bigl\lvert\sum_{n = p}^{q - 1}\mathbold{A}_n(b_n - b_{n + 1}) +
      \mathbold{A}_qb_q - \mathbold{A}_{p - 1}b_p\Bigr\rvert\\
    & \leq M\Bigl(\sum_{n = p}^{q - 1}\lvert b_n - b_{n + 1}\rvert + b_q +
      b_p\Bigr)\\
    & \leq 2Mb_p\\
    & \leq \epsilon
  \end{align*}
  The partial sums form a Cauchy sequence.
  Thus, \(\sum b_n\mathbold{a}_n\) converges.
  \par\smallskip
  \textbf{Theorem} \(\mathbold{3.45}\): If \(\sum\mathbold{a}_n\) converges
  absolutely, then \(\sum\mathbold{a}_n\) converges.
  \par\smallskip
  Let \(c_n = \lvert\mathbold{a}_n\rvert\).
  Then by theorem \(3.25\), \(\sum\mathbold{a}_n\) converges.
  \par\smallskip
  \textbf{Theorem} \(\mathbold{3.47}\): If
  \(\sum\mathbold{a}_n = \mathbold{A}\) and
  \(\sum\mathbold{b}_n = \mathbold{B}\), then
  \(\sum\mathbold{a_n + b_n} = \mathbold{A + B}\) and
  \(\sum c\mathbold{a}_n = c\mathbold{A}\) for any fixed \(c\).
  \par\smallskip
  By the previous theorems, we know that if for each component, the theorem
  holds, then the theorem holds for the vector itself.
  \par\smallskip
  \textbf{Theorem} \(\mathbold{3.55}\): If \(\sum\mathbold{a}_n\) is a series
  of vectors which converges absolutely, then every rearrangement of
  \(\sum\mathbold{a}_n\) converges, and they all converge to the same sum.
  \par\smallskip
\item
  \label{3.16}
  Fix a positive number \(\alpha\).
  Choose \(x_1 > \sqrt{\alpha}\), and define \(x_1,x_2,\ldots,\) by the
  recursive formula
  \[
  x_{n + 1} = \frac{1}{2}\Bigl(x_n + \frac{\alpha}{x_n}\Bigr).
  \]
  \begin{exercise}[label = (\alph*)]
  \item
    Prove that \(\{x_n\}\) decrease monotonically and the
    \(\lim x_n = \sqrt{\alpha}\).
    \par\smallskip
    If \(\{x_n\}\) decrease monotnically, then \(x_n - x_{n + 1} > 0\) for all
    \(n\).
    \begin{align*}
      x_n - x_{n + 1} & = x_n -
                        \frac{1}{2}\Bigl(x_n + \frac{\alpha}{x_n}\Bigr)\\
                      & = \frac{x_n^2 - \alpha}{2x_n}\eqnumtag\label{3.16a}
    \end{align*}
    Suppose, on the contrary, that \cref{3.16a} is less than zero.
    Then \(x_n < \sqrt{\alpha}\) which contradicts the fact that
    \(x_1 > \sqrt{\alpha}\).
    Thus, \cref{3.16a} is less then zero and \(\{x_n\}\) decreases
    monotonically.
    Now, \(\{x_n\}\) is bounded above by \(\sqrt{\alpha}\) and below by zero so
    \(\{x_n\}\) converges.
    Suppose the limit is \(x\).
    Then
    \begin{align*}
      \lim_{n\to\infty}x_n & =
                             \frac{1}{2}\Bigl(x_n + \frac{\alpha}{x_n}\Bigr)\\
      x & = \frac{1}{2}\Bigl(x + \frac{\alpha}{x}\Bigr)\\
      x^2 & = \alpha\\
      x & = \sqrt{\alpha}
    \end{align*}
    Thus, \(\{x_n\}\to\sqrt{\alpha}\).
  \item
    Put \(\epsilon_n = x_n - \sqrt{\alpha}\), and show that
    \[
    \epsilon_{n + 1} = \frac{\epsilon_n^2}{2x_n} <
    \frac{\epsilon_n^2}{2\sqrt{\alpha}}
    \]
    so that, setting \(\beta = 2\sqrt{\alpha}\),
    \[
    \epsilon_{n + 1} < \beta\Bigl(\frac{\epsilon_1}{\beta}\Bigr)^{2^n}
    \]
    for \(n = 1,2,\ldots\).
    \par\smallskip
    Let \(\epsilon_{n + 1} = x_{n + 1} - \sqrt{\alpha}\).
    Then
    \begin{align*}
      x_{n + 1} - \sqrt{\alpha}
      & = \frac{1}{2}\Bigl(x_n + \frac{\alpha}{x_n}\Bigr) - \sqrt{\alpha}\\
      & = \frac{x_n^2 - 2x_n\sqrt{\alpha} + \alpha}{2x_n}\\
      & = \frac{(x_n - \sqrt{\alpha})^2}{2x_n}\\
      & = \frac{\epsilon_n^2}{2x_n}
    \end{align*}
    Since \(x_n > \sqrt{\alpha}\), \(1/x_n < 1/\sqrt{\alpha}\).
    Therefore,
    \[
    \epsilon_{n + 1} = \frac{\epsilon_n^2}{2x_n} <
    \frac{\epsilon_n^2}{2\sqrt{\alpha}}\eqnumtag\label{3.16b}
    \]
    From \cref{3.16b}, we have that \(\epsilon_{n + 1} < \epsilon_n^2/\beta\).
    For \(n = 1\), we obtain
    \[
    \epsilon_2 < \frac{\epsilon_1^2}{\beta},
    \]
    and when \(n = 2\), we have
    \[
    \epsilon_3 < \frac{\epsilon_2^2}{\beta} <
    \frac{\epsilon_1^4}{\beta^2\beta} =
    \beta\Bigl(\frac{\epsilon_1}{\beta}\Bigr)^4 =
    \beta\Bigl(\frac{\epsilon_1}{\beta}\Bigr)^{2^{3 - 1}}.
    \]
    Assume this is true for \(k < n\).
    Then \(\epsilon_k < \beta(\epsilon_1/\beta)^{2^{k - 1}}\).
    \[
    \epsilon_{k + 1} < \frac{\epsilon_k^2}{\beta} < \frac{\beta^2}{\beta}
    \biggl(\frac{\epsilon_1^{2^{k - 1}}}{\beta^{2^{k - 1}}}\biggr)^2 =
    \beta\Bigl(\frac{\epsilon_1}{\beta}\Bigr)^{2^k}
    \]
    By the principle of mathematical induction,
    \(\epsilon_{n + 1} = \beta(\epsilon_1/\beta)^{2^n}\) for
    \(n\in\mathbb{Z}^+\).
  \item
    This is a good algorithm for computing square roots, since the recursion
    formula is simple and the convergence is extremely rapid.
    For example, if \(\alpha = 3\) and \(x_1 = 2\), show that
    \(\epsilon_1/\beta < \frac{1}{10}\) and that therefore
    \[
    \epsilon_5 < 4\cdot 10^{-16},\qquad\epsilon_6 < 4\cdot 10^{-23}.
    \]
    \(\epsilon_1 = x_1 - \sqrt{\alpha}\) and \(\beta = 2\sqrt{\alpha}\) so
    \[
    \frac{\epsilon_1}{\beta} = \frac{2 - \sqrt{3}}{2\sqrt{3}}\approx 0.077 <
    \frac{1}{10}
    \]
    For \(\epsilon_5\) and \(\epsilon_6\), we have
    \begin{align*}
      \epsilon_5 & < \beta\Bigl(\frac{\epsilon_1}{\beta}\Bigr)^{2^k}\\
                 & \approx 5.69\times 10^{-18}\\
                 & < 4\cdot 10^{-16}\\
      \epsilon_6 & < \beta\Bigl(\frac{\epsilon_1}{\beta}\Bigr)^{2^k}\\
                 & \approx 9.34\times 10^{-36}\\
                 & < 4\cdot 10^{-23}
    \end{align*}
  \end{exercise}
\item
  Fix \(\alpha > 1\).
  Take \(x_1 > \sqrt{\alpha}\), and define
  \[
  x_{n + 1} = \frac{\alpha + x_n}{1 + x_n} = x_n +
  \frac{\alpha - x_n^2}{1 + x_n}.
  \]
  \begin{exercise}[label = (\alph*)]
  \item
    Prove that \(x_1 > x_3 > x_5 > \cdots\).
  \item
    Prove that \(x_2 < x_4 < x_6 < \cdots\).
  \item
    Prove that \(\lim x_n = \sqrt{\alpha}\).
  \item
    Compare the rapidity of convergence of this process with the one described
    in \cref{3.16}.
  \end{exercise}
\item
  Replace the recursion formula of \cref{3.16} by
  \[
  x_{n + 1} = \frac{p - 1}{p}x_n + \frac{\alpha}{p}x_n^{-p + 1}
  \]
  where \(p\) is a fixed positive integer, and describe the behavior of the
  resulting sequences \(\{x_n\}\).
\item
  Associate to each sequence \(a = \{\alpha_n\}\), in which \(\alpha_n\) is
  \(0\) or \(2\), the real number
  \[
  x(a) = \sum_{n = 1}^{\infty}\frac{\alpha_n}{3^n}.
  \]
  Prove that the set of all \(x(a)\) is precisely the Cantor set described in
  section \(2.44\).
  \par\smallskip
  The cantor set is constructed by taking a segment of unit length \([0,1]\)
  and removing the middle third, \((1/3,2/3)\).
  That is, we are left with \([0,1/3] \cup [2/3,1]\) and then doing this ad
  infinitum.
  Let's consider \(1/3\) in base three (ternary).
  \[
  \frac{1}{3} = 0\cdot 3^1 + 1\cdot\frac{1}{3} + 0\cdot
  \sum_{n = 2}^{\infty}\frac{1}{3^n} = 0.1
  \]
  Well this poses a problem since \(\{\alpha_n\}\) is a sequence of \(0\) or
  \(2\).
  Suppose we can write \(0.1\) as \(0.0\bar{2}\) instead. 
  \[
  0.0\bar{2} = 2\Bigl(\frac{1}{3^2} + \frac{1}{3^3} + \cdots\Bigr) =
  2\sum_{n = 2}^{\infty}\frac{1}{3^n} =
  \frac{2}{9}\sum_{n = 0}^{\infty}\frac{1}{3^n} = \frac{2}{9}\frac{1}{1-1/3} =
  \frac{1}{3} = 0.1\eqnumtag\label{3.19}
  \]
  Now, let's go back to the Cantor set.
  After the first iteration, in ternary, we removed \((0.1, 0.2)\).
  That is, we removed all the terms with \(0.1\ldots\) as the first digit but
  it appears that we kept \(0.1\).
  From \cref{3.19}, we see that we actually keep
  \([0, 0.0\bar{2}] \cup [0.2, 0.\bar{2}]\) where \(1 = 0.\bar{2}\) be the
  same argument.
  With the second step, we remove all the digits with \(0.01\) as the first
  digit and keep \(0.01 = 0.00\bar{2}\).
  Since this continues ad infinitum, we are left with a set that is
  represented by the ternary expansion
  \[
  \sum_{n = 1}^{\infty}\frac{\alpha_n}{3^n}
  \]
  where \(\{\alpha_n\}\) contains only \(0\) and \(2\) which is the Cantor set.
\item
  Suppose \(\{p_n\}\) is a Cauchy sequence in a metric space \(X\), and some
  subsequence \(\{p_{n_i}\}\) converges to a point \(p\in X\).
  Prove that the full sequence \(\{p_n\}\) converges to \(p\).
  \par\smallskip
  Since \(\{p_{n_i}\}\) converges, \(\{p_{n_i}\}\) is Cauchy.
  That is, there exist \(n,m\) such that for \(n,m > N\),
  \begin{align*}
    \lvert p_{n_i} - p_{m_i}\rvert
    & = \lvert p_{n_i} - p - (p_{m_i} - p)\rvert\\
    & \leq \lvert p_{n_i} - p\rvert + \lvert p_{m_i} - p\rvert
  \end{align*}
  Let \(\epsilon > 0\) be given.
  Now, the subsequence \(\{p_{n_i}\}\) converges so for \(n > N_1\),
  \(\lvert p_{n_i} - p\rvert < \epsilon/2\).
  Take \(n,m > \max\{N, N_1\}\) then
  \[
  \lvert p_{n_i} - p_{m_i}\rvert < \frac{\epsilon}{2} + \frac{\epsilon}{2}
  = \epsilon.
  \]
  Therefore, every subsequences converges to \(p\) and \(\{p_n\}\to p\) as
  well.
\item
  \label{3.21}
  Prove the following analogue of Theorem \(3.10(b)\): If \(\{E_n\}\) is a
  sequence of closed nonempty and bounded sets in a \textit{complete} metric
  space \(X\), if \(E_n\supset E_{n + 1}\), and if
  \[
  \lim_{n\to\infty}\operatorname{diam} E_n = 0,
  \]
  then \(\bigcap_1^{\infty}E_n\) consists of exactly one point.
\item
  Suppose \(X\) is a nonempty complete metric space, and \(\{G_n\}\) is a
  sequence of dense open subsets of \(X\).
  Prove Baire's theorem, namely, that \(\bigcap_1^{\infty}G_n\) is not empty.
  (In fact, it is dense in \(X\).)
  \textit{Hint: Find a shrinking sequence of neighborhoods \(E_n\) such that
    \(\bar{E}_n\subset G_n\), and apply \cref{3.21}.}
\item
  \label{3.23}
  Suppose \(\{p_n\}\) and \(\{q_n\}\) are Cauchy sequences in a metric space
  \(X\).
  Show that the sequence \(\{d(p_n,q_n)\}\) converges.
  \textit{Hint: For any \(m,n\)
    \[
    d(p_n, q_n)\leq d(p_n, p_m) + d(p_m, q_m) + d(q_m, q_n);
    \]
    it follows that
    \[
    \lvert d(p_n, q_n) - d(p_m, q_m)\rvert
    \]
    is small if \(m\) and \(n\) are large.}
  \par\smallskip
  Let \(\epsilon > 0\) be given.
  Since \(\{p_n\}\) and \(\{q_n\}\) are Cauchy, there exist
  \(n,m > N = \max\{N_1,N_2\}\) such that
  \[
  \lvert p_n - p_m\rvert < \frac{\epsilon}{2}\qquad\text{and}\qquad
  \lvert q_n - q_m\rvert < \frac{\epsilon}{2}
  \]
  simultaneously.
  Then
  \begin{align*}
    d(p_n, q_n) & = \lvert p_n - q_m\rvert\\
                & = \lvert p_n - p_m + p_m - q_m + q_m - q_n\rvert\\
                & \leq \lvert p_n - p_m\rvert + \lvert p_m - q_m\rvert +
                  \lvert q_n - q_m\rvert\\
    d(p_n, q_n) - d(p_m, q_m) & < \epsilon\\
    \lvert d(p_n, q_n) - d(p_m, q_m)\rvert & < \epsilon
  \end{align*}
  Therefore, \(\{d(p_n,q_n)\}\) is Cauchy and hence converges.
\item
  Let \(X\) be a metric space.
  \begin{exercise}[label = (\alph*), ref = \arabic{exercisei} (\alph*)]
  \item
    Call two Cauchy sequences \(\{p_n\},\{q_n\}\) in \(X\) \textit{equivalent}
    if
    \[
    \lim_{n\to\infty} d(p_n,q_n) = 0.
    \]
    Prove that this is an equivalence relation.
    \par\smallskip
    Since \(d\) is a distance function on the metric space \(X\), \(d\)
    satisfies the metric properties.
    Then \(d(p_n,p_n) = 0\) for all \(n\) since \(d(x,y) = 0\) if \(x = y\).
    Thus, \(\lim_{n\to\infty}d(p_n,q_n)\) reflexive.
    The second metric property is that of being symmetric; hence,
    \(\lim_{n\to\infty}d(p_n,q_n)\) is symmetric.
    Suppose \(\{p_n\}\sim\{q_n\}\) and \(\{q_n\}\sim\{t_n\}\).
    Then \(\lim_{n\to\infty}d(p_n,q_n)\) and \(\lim_{n\to\infty}d(q_n,t_n)\)
    are both zero.
    Now
    \begin{align*}
      \lim_{n\to\infty}d(p_n, t_n) & \leq \lim_{n\to\infty}d(p_n, q_n) +
                                     \lim_{n\to\infty}d(q_n, t_n)\\
      \intertext{by the triangle inequality.}
      \lim_{n\to\infty}d(p_n, t_n) & \leq 0
    \end{align*}
    Since \(d(x,y)\geq 0\), \(\lim_{n\to\infty}d(p_n, t_n) = 0\).
    Thus, \(\lim_{n\to\infty}d(p_n,q_n)\) is transitive.
  \item
    Let \(X^*\) be the set of all equivalence classes so obtained .
    If \(P,Q\in X^*\), \(\{p_n\}\in P\), \(\{q_n\}\in Q\), define
    \[
    \Delta(P,Q) = \lim_{n\to\infty} d(p_n, q_n);
    \]
    by \cref{3.23}, this limit exists.
    Show that the number \(\Delta(P,Q)\) is unchanged if \(\{p_n\}\) and
    \(\{q_n\}\) are replaced by equivalent sequences, and hence \(\Delta\)
    is a distance function in \(X^*\).
    \par\smallskip
    Let \(\{p_n'\}\) and \(\{q_n'\}\) be equivalent to \(\{p_n\}\) and
    \(\{q_n\}\), respectively.
    By \cref{3.23}, we have that
    \begin{align*}
      \lim_{n\to\infty}d(p_n', q_n')
      & \leq \lim_{n\to\infty}d(p_n', p_n) + \lim_{n\to\infty}d(p_n, q_n) +
        \lim_{n\to\infty}d(q_n', q_n)\\
      & = \lim_{n\to\infty}d(p_n, q_n)\\
      & \leq \lim_{n\to\infty}d(p_n, p_n') + \lim_{n\to\infty}d(p_n', q_n') +
        \lim_{n\to\infty}d(q_n, q_n')\\
      & = \lim_{n\to\infty}d(p_n', q_n')
    \end{align*}
    Therefore, \(\lim_{n\to\infty}d(p_n',q_n') = \lim_{n\to\infty}d(p_n,q_n)\)
    since
    \[
    \lim_{n\to\infty}d(p_n', q_n')\leq\lim_{n\to\infty}d(p_n, q_n)\leq
    \lim_{n\to\infty}d(p_n', q_n').
    \]
  \item
    Prove that the resulting metric space \(X^*\) is complete.
  \item
    \label{3.24d}
    For each \(p\in X\), there is a Cauchy sequence all of whose terms are
    \(p\); let \(P_p\) be the element of \(X^*\) which contains this sequence.
    Prove that
    \[
    \Delta(P_p, P_q) = d(p, q)
    \]
    for all \(p,q\in X\).
    In other words, the mapping \(\varphi\) defined by \(\varphi(p) = P_p\) is
    an isometry (that is, a distance-preserving mapping) of \(X\) into \(X^*\).
  \item
    Prove that \(\varphi(X)\) is dence in \(X^*\), and that
    \(\varphi(X) = X^*\) if \(X\) is complete.
    By \cref{3.24d}, we may identify \(X\) and \(\varphi(X)\) and thus regard
    \(X\) as embedded in the complete metric space \(X^*\).
    We call \(X^*\) the \textit{completion} of \(X\).
  \end{exercise}
\item
  Let \(X\) be the metric space whose points are the rational numbers, with the
  metric \(d(x,y) = \lvert x - y\rvert\).
  What is the completion of this space?
\end{exercise}
%%% Local Variables:
%%% mode: latex
%%% TeX-master: t
%%% End:
