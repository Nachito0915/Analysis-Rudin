\chapter{Basic Topology}
\label{ch2}

\begin{exercise}
\item
  Prove that the empty set is a subset of every set.
  \par\smallskip
  Let \(A\) be set.
  If \(x\not\in A\), then \(x\not\in\varnothing\).
  Since \(\varnothing\) is the empty set, \(x\not\in\varnothing\) is a given.
  By contrapositive, if \(x\in\varnothing\), then \(x\in A\); therefore,
  \(\varnothing\subset A\).
\item
  A complex number \(z\) is said to be \textit{algebraic} if there are integers
  \(a_0,\ldots,a_n\), not all zero, such that
  \[
  a_0z^n + a_1z^{n - 1} + \cdots + a_{n - 1}z + a_n = 0.
  \]
  Prove that the set of all algebraic numbers is countable.
  \textit{Hint: For every positive integer \(N\) there are only finitely many
    equations with}
  \[
  n + \lvert a_0\rvert + \lvert a_1\rvert + \cdots + \lvert a_n\rvert = N.
  \]
  Let \(N\in\mathbb{Z}^+\) and \(A_N\) be the set of algebraic equations for a
  given \(N\).
  Since \(1\leq n\leq N\), each \(A_N\) is finite.
  The set of algebraic numbers is \(\bigcup_{N\in\mathbb{Z}^+}A_n\).
  The union of countable sets is countable so the set of algebraic numbers is
  countable.
\item
  Prove that there exist real numbers which are not algebraic.
  \par\smallskip
  The set of algebraic numbers are countable.
  Therefore, the set of algebraic real numbers would also be countable.
  The real numbers are an uncountable set and the union of uncountable sets are
  not countable. We have reached a contradiction so there are real numbers which
  are not algebraic.
\item
  Is the set of all irrational real numbers countable?
  \par\smallskip
  No.
  Let \(\mathbb{I}\) be the set of irrational numbers and \(\mathbb{Q}\) be the
  set of rational numbers.
  Then \(\mathbb{R} = \mathbb{I}\cup\mathbb{Q}\).
  The set of rational numbers is countable.
  % prove Q is countable
  If \(\mathbb{I}\) were countable, then \(\mathbb{R}\) would be countable as
  well.
\item
  Construct a bounded set of real numbers with exactly three limit points.
  \par\smallskip
  Let \(A_0 = \{1/n\mid n\in\mathbb{Z}^+\}\),
  \(A_1 = \{1 + 1/n\mid n\in\mathbb{Z}^+\}\), and
  \(A_2 = \{2 + 1/n\mid n\in\mathbb{Z}^+\}\).
  Then the limit point of \(A_0\) is \(0\), the limit point of \(A_1\) is
  \(1\), and the limit point of \(A_2\) is \(2\).
  Let \(S = A_0\cup A_1\cup A_2\).
  Now \(S\) is bounded below by zero and above by three with limit points
  \(0,1,2\).
\item
  Let \(E'\) be the set of all limit points of a set \(E\).
  Prove that \(E'\) is closed.
  Prove that \(E\) and \(\bar{E}\) have the same limit points.
  (Recall that \(\bar{E} = E\cup E'\).)
  Do \(E\) and \(E'\) always have the same limit points?
  \par\smallskip
  Let \(x\not\in E'\).
  Then \(x\) is not a limit point of \(E\).
  Now \(x\) has a neighborhood which doesn't intersect with \(E'\) so the
  complement of \(E'\) is open; therefore, \(E'\) is closed.
  If \(x\) is a limit point of \(E\) then \(x\in E'\) so \(x\) is a limit point
  of \(\bar{E}\).
  Suppose \(x\) is a limit point of \(\bar{E}\).
  Then \(x\in\bar{E}\) since \(\bar{E}\) is closed.
  Thus, \(x\in E'\) or \(x\in E\).
  If \(x\in E'\), then \(x\) is a limit point of \(E\) so suppose \(x\) is in
  \(E\).
  Then we have a neighborhood \(N_r(x)\) for \(r > 0\) such that
  \(N\cap E = \{x\}\).
  Since \(E'\) is closed, \(x\) isn't a limit point of \(E'\).
  Let \(M_r(x)\) be a neighborhood of \(x\) such that
  \(M\cap E' = \varnothing\).
  Let \(V = N\cap M\) where \(V\) is a neighborhood of \(x\).
  Now
  \(V\cap\bar{E} = (V\cap E)\cup (V\cap E') = \{x\}\cup\varnothing = \{x\}\); therefore,
  \(x\) is a limit point of \(\bar{E}\) so \(x\in E'\) and is a limit point
  of \(E\).
  No, consider \(E = \{\frac{1}{n}\mid n\in\mathbb{Z}^+\}\).
  Then \(E' = \{0\}\) and \((E')^c\) is \(\varnothing\).
\item
  Let \(A_1, A_2, \ldots\) be subsets of a metric space.
  \begin{exercise}[label = (\alph*), ref = \arabic{exercisei} (\alph*)]
  \item
    \label{1.7.a}
    If \(B_n = \bigcup_{i = 1}^nA_i\), prove that
    \(\bar{B}_n = \bigcup_{i = 1}^n\bar{A}_i\) for \(n = 1, 2, \ldots\)
    \par\smallskip
    For \(n = 2\), \(\bar{B} = \overline{A_1\cup A_2} =
    \bar{A}_1\cup \bar{A}_2\).
    Suppose \(x\in\overline{A_1 \cup A_2}\).
    Then \(x\in A_1\cup A_1'\cup A_2\cup A_2'\) since \(\bar{E} = E\cup E'\).
    Therefore, \(x\in\bar{A}_1\cup\bar{A}_2\) so
    \(\overline{A_1\cup A_2}\subseteq\bar{A}_1\cup \bar{A}_2\).
    Suppose \(x\in\bar{A}_1\cup \bar{A}_2\).
    Then \(x\in A_1\cup A_2\cup (A_1\cup A_2)' = \overline{A_1\cup A_2}\).
    Thus, we have that
    \(\bar{A}_1\cup \bar{A}_2\subseteq\overline{A_1\cup A_2}\) and that
    \(\bar{A}_1\cup \bar{A}_2 = \overline{A_1\cup A_2}\).
    Now we can show the closure of the union of \(n\) subsets is the union of
    closure of the subsets.
    \begin{align*}
      \bar{B}_n & = \overline{\bigcup_{i = 1}^nA_i}\\
                & = \overline{A_1\cup\bigcup_{i = 2}^nA_i}\\
                & = \bar{A}_1\cup\bigcup_{i = 2}^n\bar{A}_i\\
                & = \bigcup_{i = 1}^n\bar{A}_i
    \end{align*}
  \item
    If \(B_n = \bigcup_{i = 1}^{\infty}A_i\), prove that
    \(\bar{B}\supset\bigcup_{i = 1}^{\infty}\bar{A}_i\).
    \par\smallskip
    From the premise, we have that \(B_n\subseteq\bigcup_{i = 1}^{\infty}A_i\)
    and \(B_n\supseteq\bigcup_{i = 1}^{\infty}A_i\).
  \end{exercise}
\item
  Is every point of every open set \(E\subset\mathbb{R}^2\) a limit point of
  \(E\)?
  Answer the same question for closed sets of \(\mathbb{R}^2\).
  \par\smallskip
  Let \(x\in E\).
  Let \(\epsilon > 0\) be given.
  Let \(N_{\epsilon}(x)\) be a neighborhood about \(x\) of radius \(\epsilon\).
  Now \(N\cap E\subset\mathbb{R}^2\) and the intersection of a finite number of
  open sets is open.
  Therefore, \(N\cap E\) open neighborhood about \(x\).
  Thus, \(x\) is a limit point of \(E\).
  \par\smallskip
  Let the closed set \(E\) consist of only the point \(p = (0, 0)\).
  Every open neighborhood of \(p\) contains no points of \(E\) except \(p\).
  Thus, \(p\) is not a limit point of \(E\).
\item
  Let \(E^{\circ}\) denote the set of all interior points of a set \(E\).
  \begin{exercise}[label = (\alph*), ref = \arabic{exercisei}(\alph*)]
  \item
    Prove that \(E^{\circ}\) is always open.
    \par\smallskip
    Let \(x\in E^{\circ}\) and \(\epsilon > 0\).
    There exists \(y\in E\) such that \(d(x,y) < \epsilon\).
    Let \(r = \epsilon - d(x,y) > 0\).
    If \(d(z,y) < r = \epsilon - d(x,y)\), then \(d(z,y) + d(x,y) < \epsilon\).
    By the triangle inequality, \(d(x,z)\leq d(z,y) + d(x,y) < \epsilon\).
    Therefore, \(z\in E\).
    Now, \(y\in E^{\circ}\) if there is a neighborhood of \(y\) such that
    \(N_{\delta}(y)\subset E\).
    Let \(\delta < \epsilon/2\).
    Then \(N_{\delta}(y) = d(x,y)\subset E\).
    Thus, \(y\) is interior point and \(E^{\circ}\) is always open.
  \item
    \label{2.9b}
    Prove that \(E\) is open if and only if \(E^{\circ} = E\).
    \par\smallskip
    Suppose \(E\) is open.
    By definition, \(E\) is open if every point of \(E\) is an interior point
    of \(E\) or \(E = E^{\circ}\).
    Suppose \(E = E^{\circ}\).
    Then every point of \(E\) is an interior point of \(E\) so it follows that
    \(E\) is open by definition.
  \item
    If \(G\subset E\) and \(G\) is open, prove that \(G\subset E^{\circ}\).
    \par\smallskip
    By \cref{2.9b}, if \(G\) is open, \(G = G^{\circ}\).
    Therefore, \(G = G^{\circ}\subseteq E^{\circ}\subset E\) as was needed to
    be shown.
  \item
    Prove that the complement of \(E^{\circ}\) is the closure of the
    complement of \(E\).
  \item
    Do \(E\) and \(\bar{E}\) always have the same interiors?
    \par\smallskip
    Let \(E = \mathbb{Q}\).
    Then \(\bar{E} = \mathbb{R}\) so \(E^{\circ} = \varnothing\) and
    \(\bar{E}^{\circ} = \mathbb{R}\).
    Thus, they don't always have the same interiors.
  \item
    Do \(E\) and \(E^{\circ}\) always have the same closures?
    \par\smallskip
    Let \(E = \mathbb{Q}\).
    Then \(E^{\circ} = \varnothing\) so \(\bar{E} = \mathbb{R}\) and
    \(\bar{E^{\circ}} = \varnothing\).
    Thus, they don't always have the same closures.
  \end{exercise}
\item
  Let \(X\) be an infinite set.
  For \(p\in X\) and \(q\in X\), define
  \[
  d(p, q) =
  \begin{cases}
    1, & \text{if } p\neq q\\
    0, & \text{if } p = q
  \end{cases}
  \]
  Prove that this a metric space.
  Which subsets of the resulting metric space are open?
  Which are closed? Which are compact?
  \par\smallskip
  By definition, the separation and coincidence axioms are satisfied.
  That is,
  \[
  d(p, q)\geq 0
  \]
  for \(p\neq q\) and zero when \(p = q\).
  For \(p\neq q\), \(d(p,q) = 1 = d(q,p)\), and when \(p = q\),
  \(d(p,q) = d(p,p) = 0\).
  Thus, symmetry is satisfied \(d(p,q) = d(q,p)\).
  For the triangle inequality, if \(p = q = r\), then we have
  \(d(p,q)\leq d(p,r) + d(q,r)\Rightarrow 0\leq 0\).
  If \(p\neq q\neq r\), then \(1\leq 2\), and if \(p = q\), then \(0\leq 2\).
\item
  For \(x\in\mathbb{R}\) and \(y\in\mathbb{R}\), define
  \begin{align*}
    d_1(x, y) & = (x - y)^2\\
    d_2(x, y) & = \sqrt{\lvert x - y\rvert}\\
    d_3(x, y) & = \lvert x^2 - y^2\rvert\\
    d_4(x, y) & = \lvert x - 2y\rvert\\
    d_5(x, y) & = \frac{\lvert x - y\rvert}{1 + \lvert x - y\rvert}
  \end{align*}
  Determine for each of these, whether it is a metric or not.
  \par\smallskip
  For \(d_1\), note that squaring is \(\geq 0\) for all \(x,y\in\mathbb{R}\)
  and only when \(x = y\Rightarrow (x - x)^2 = 0\).
  For symmetry, it is easy to show that
  \((x - y)^2 = [(-1)(y - x)]^2 = (y - x)^2\).
  For the triangle inequality, assume that \(x\neq y\neq z\) because if they
  are we have \(0\leq 0\) and the identity holds.
  \begin{align*}
    d_1(x, z) & \leq d_1(x, y) + d_1(z, y)\\
    x^2 - 2xz + z^2 & \leq x^2 - 2xy + 2y^2 -2yz + z^2\\
    y(x + z) & \leq y^2 + xz
  \end{align*}
  Take \(y = 0\).
  Then \(0\leq xz\).
  As long as either \(x\) or \(y\) are different signs \(\pm\), the inequality
  doesn't hold.
  For instance, let \(y = 0\), \(x = -1\), and \(z = 2\).
  \[
  9\not\leq 1 + 4 = 5
  \]
  Therefore, \(d_1\) is not a metric.
  For \(d_2\), \(\lvert x - y\rvert\geq 0\) and \(\lvert x - y\rvert = 0\) iff
  \(x = y\).
  Therefore, \(d_2(x,y) = \sqrt{\lvert x - y\rvert}\geq 0\) and zero iff
  \(x = y\).
  \[
  d_2(x, y) = \sqrt{\lvert x - y\rvert} = \sqrt{\lvert (-1)(y - x)\rvert} =
  \sqrt{\lvert y - x\rvert} = d_2(y, x)
  \]
  For the triangle inequality, it is vacuously true when \(x = y = z\).
  \begin{align*}
    d_2(x, z) & \leq d_2(x, y) + d_2(y, z)\\
    \lvert x - z\rvert & \leq \lvert x - y\rvert + \lvert y - z\rvert +
                         2\sqrt{\lvert x - y\rvert}\sqrt{\lvert y - z\rvert}
  \end{align*}
  By the triangle inequality, we can write \(\lvert x - z\rvert\) as
  \[
  \lvert x - z\rvert = \lvert x - y + y - z\rvert\leq\lvert x - y\rvert +
  \lvert y - z\rvert
  \]
  Since \(2\sqrt{\lvert x - y\rvert}\sqrt{\lvert y - z\rvert} > 0\) for
  \(x\neq y\neq z\), \(d_2(x,z)\leq d_2(x,y) + d_2(y,z)\) and \(d_2\) is a
  metric.
  \(d_3\) is not a metric since \(\lvert x^2 - y^2\rvert = 0\) if \(x = -y\) or
  \(y = -x\).
  For example, let \(x = 1\) and \(y = -1\).
  Then \(\lvert 1 - (-1)^2\rvert = 0\).
  \(d_4\) is not metric since \(\lvert x - 2x\rvert = \lvert -x\rvert =
  \lvert x\rvert\) which is only zero when \(x = 0\).
  Therefore, for all \(x,y\in\mathbb{R}\), \(x = y\) doesn't yield zero.
  For \(d_5\), we have already established that \(\lvert x - y\rvert\geq 0\)
  and zero iff \(x = y\).
  Since the numerator is \(\lvert x - y\lvert\),
  \[
  \frac{\lvert x - y\rvert}{1 + \lvert x - y\rvert}\geq 0
  \]
  and zero iff \(x = y\).
  \begin{align*}
    d_5(x, y) & = \frac{\lvert x - y\rvert}{1 + \lvert x - y\rvert}\\
              & = \frac{\lvert (-1)(y - x)\rvert}
                {1 + \lvert (-1)(y - x)\rvert}\\
              & = \frac{\lvert y - x\rvert}{1 + \lvert y - x\rvert}\\
              & = d_5(y, x)
  \end{align*}
  For the triangle inequality, we will multiple through by
  \((1 + \lvert x - z\rvert)(1 + \lvert x - y\rvert)(1 + \lvert y - z\rvert)\).
  After simplifying, we will be left with
  \[
  \lvert x - z\rvert\leq\lvert x - y\rvert + \lvert y - z\rvert +
  2\lvert x - y\rvert\lvert y - z\rvert + \lvert x - y\rvert\lvert y - z\rvert
  \lvert x - z\rvert
  \]
  By the triangle inequality, we have that
  \(\lvert x - z\rvert\leq\lvert x - y\rvert + \lvert y - z\rvert\).
  Since the other terms are strictly greater than or equal to zero with
  equality only when \(x = y = z\), we can see that \(d_5\) is a metric.
\item
  Let \(K\subset\mathbb{R}\) consist of \(0\) and the numbers \(1/n\) for
  \(n = 1,2,\ldots\)
  Prove that \(K\) is compact directly from the definition (without using the
  Heine-Borel theorem).
  \par\smallskip
  Let \(\{G_{\alpha}\}\) be an open cover of \(K\).
  Therefore, for each \(\alpha\), \(G_{\alpha}\) is open.
  Now, \(0\in K\subset\cup_{\alpha}G_{\alpha}\) so \(0\) exists in some
  \(G_{\alpha}\).
  Let \(0\in G_{\alpha_0}\).
  For each \(G_{\alpha}\), we can find a neighborhood of \(0\) contained in
  \(G_{\alpha}\).
  Let \(\epsilon > 0\).
  Then \(N_{\epsilon}(0)\subset G_{\alpha}\).
  Let \(n\geq N\) such that \(n > 1/\epsilon\iff\epsilon > 1/n\).
  Since \(n < n + 1 < \cdots\), \(\epsilon > 1/n > 1/(n + 1) > \cdots\).
  We can look at this in terms of a metric \(d\)
  \[
  \epsilon > d(0, 1/n) > d[0, 1/(n + 1)] > \cdots
  \]
  For all \(n\geq N\), \(1/n\in N_{\epsilon}(0)\subset G_{\alpha_0}\).
  The only points not in \(G_{\alpha_0}\) are \(\{1,1/2,\ldots,1/(n - 1)\}\).
  Since each of these points are in \(K\subset G_{\alpha}\), each point exists
  in some \(G_{\alpha_i}\).
  Let \(1\in G_{\alpha_1}\), \(1/2\in G_{\alpha_2},\ldots,\)
  \(G_{\alpha_{n - 1}}\).
  Thus, \(K\) belongs to a finite cover
  \(K\subset\bigcup_{i = 0}^{n - 1}G_{\alpha_i}\).
\item
  Construct a compact set of real numbers whose limit points form a countable
  set.
  \par\smallskip
  Let \(A = \{0\}\cup\{1/n\colon n\in\mathbb{Z}^+\}\cup
  \{1/n + 1/m\colon n,m\in\mathbb{Z}^+\}\).
  Then \(A' = \{0\}\cup\{1/n\colon n\in\mathbb{Z}^+\}\).
  Since for each \(n\) and \(m\to\infty\), \(\{1/n + 1/m\}\to 1/n\) and
  for \(n,m\to\infty\), \(\{1/n + 1/m\}\to 0\).
  Thus, \(A'\subset A\) and \(A\) is closed.
  Since \(n,m\in\mathbb{Z}^+\), the elements of \(A\) are all non-negative so
  \(A\) is bounded below by zero.
  The maximum of \(A\) is when \(n = m = 1\) which is \(\max\{A\} = 2\);
  therefore, \(A\) is bounded above by two.
  Now, \(A\) is closed and bounded so by the Heine-Borel theorem, \(A\) is
  compact.
  The limit points of \(A\) are the non-negative rational numbers of the form
  \(1/n\).
  Let \(n\in\mathbb{Z}^+\).
  Let \(x = 1/n\).
  Define
  \[
  f(x) =
  \begin{cases}
    0, & x = 0\\
    n, & x = \frac{1}{n}
  \end{cases}
  \]
  Clearly, \(f\) is a bijection.
  Now, \(f\) enumerates \(\{0,1,2,3,4,\ldots\}\) which is the set
  \(\mathbb{Z}^{\geq 0}\).
  Thus, \(f\) is a bijection to a subset of the integers; therefore, \(A\) is
  a compact set of real numbers whose limit points form a countable set.
\item
  Give an example of an open cover of the segment \((0,1)\) which has no
  finite subcover.
  \par\smallskip
  Let \(K_n = \bigcup_{n = 1}^{\infty}\bigl(\frac{1}{n},1\bigr)\).
  Then \(K_n\) is an open cover of \((0,1)\).
  Suppose there exists a finite \(N > 0\) such that \(\{K_1,\ldots,K_N\}\)
  covers \((0,1)\).
  Then \(\bigcup_{n = 1}^NK_n = \bigl(\frac{1}{N},1\bigr)\) but
  \(\frac{1}{2N}\notin\bigcup_{n = 1}^NK_n\).
  Therefore, \(K_n\) doesn't have a finite subcover of \((0,1)\).
\item
  Show that Theorem \(2.36\) and its Corollary become false (in \(\mathbb{R}\),
  for example) if the word "compact" is replaced by "closed" or by "bounded".
  \par\smallskip
  Theorem \(2.36\): If \(\{K_{\alpha}\}\) is a collection of compact subsets of
  a metric space \(X\) such that the intersection of every finite subcollection
  of \(\{K_{\alpha}\}\) is nonempty, then \(\cap K_{\alpha}\) is nonempty.
  \par\smallskip
  Consider the bounded set \(\{K_{\alpha}\} = (0,1/n)\) and the closed set
  \(\{K_{\beta}\} = [n,\infty)\).
  Let \(N > 0\) and finite.
  Then
  \begin{align*}
    \bigcap_{\alpha = 1}^NK_{\alpha} & = \Bigl(0, \frac{1}{N}\Bigr)\\
    \bigcap_{\beta = 1}^NK_{\beta} & = [N, \infty)
  \end{align*}
  which are both nonempty.
  If we take the intersection of the entire family of sets, our intersection
  will be empty since neither zero nor infinity is in their respected set.
  That is,
  \begin{align*}
    \bigcap_{\alpha = 1}^{\infty}K_{\alpha} & = \varnothing\\
    \bigcap_{\beta = 1}^{\infty}K_{\beta} & = \varnothing
  \end{align*}
  Thus, theorem \(2.36\) is false when compact is replaced by either closed or
  bounded.
  \par\smallskip
  Corollary: If \(\{K_{\alpha}\}\) is a sequence of nonempty compact sets such
  that \(K_n\supset K_{n + 1}\) for \(n=1,2,\ldots\), then
  \(\bigcap_{n = 1}^{\infty}K_n\) is not empty.
  \par\smallskip
  Consider the previous family of sets \(\{K_{\alpha}\}\) and
  \(\{K_{\beta}\}\).
\item
  Regard \(\mathbb{Q}\), the set of all rational numbers, as a metric space,
  with \(d(p, q) = \lvert p - q\rvert\).
  Let \(E\) be the set of all \(p\in\mathbb{Q}\) such that \(2 < p^2 < 3\).
  Show that \(E\) is closed and bounded in \(\mathbb{Q}\), but that \(E\) is
  not compact.
  Is \(E\) open in \(\mathbb{Q}\)?
  \par\smallskip
  The set \(E\) is then
  \(E = \bigl\{p\in\mathbb{Q}\colon\bigl(-\sqrt{3},-\sqrt{2}\bigr)\cup
  \bigl(\sqrt{2},\sqrt{3}\bigr)\text{ such that } 2 < p^2 < 3\bigr\}\).
  For all \(p\in E\), \(1 < p\) and \(p < 2\); thus, \(E\) is bounded in \(\mathbb{Q}\).
  \par\smallskip
  Let \(x\in\mathbb{Q}\) be a limit point of \(E\).
  If \(x^2 < 2\), then \(\pm x < \sqrt{2}\iff\lvert x\rvert < \sqrt{2}\iff 0 < \sqrt{2} - \lvert x\rvert = r_1\).
  Let \(z\in N_{r_1}(x)\).
  \[
  	\begin{aligned}
		\lvert z\rvert &= \lvert z + x - x\rvert\\
		&= \lvert -(x - z) + \lvert x\rvert\\
		&\leq\lvert x - z\rvert + \lvert x\rvert\\
		&< r_1 + \lvert x\rvert\\
		&= \sqrt{2}
	\end{aligned}
  \]
  Therefore, \(z^2 < 2\) and \(z\not\in E\).
  A contradiction since a neighborhood of a limit point \(E\) must contain a point of \(E\).
  Similarly, we can take \(x^2 > 3\) to show again \(z\in N_{r_2}(x)\) leads to \(z\not\in E\).
  Thus, \(E\) must contain its limits points and \(E\) is closed in \(\mathbb{Q}\).
  \par\smallskip
  Let \(\{U_n\}\) be a collection of open covers defined as
  \begin{gather}
  	U_n = \mathbb{Q}\cap\big\{(-\sqrt{3 - 1/n}, -\sqrt{2 + 1/n}) \cup (\sqrt{2 + 1/n}, \sqrt{3 - 1/n})\big\},
	\qquad n\ge 3\notag\\
	2 + \frac{1}{n} < x^2 < 3 - \frac{1}{n}\notag
  \end{gather}
  Let \(\{U_n\}_{n\in I}\) be a finite sub collection.
  Let \(N = \max_{n\in I}\{n\}\).
  We know that \(\mathbb{Q}\) is dense in \(\mathbb{R}\) from Theorem \(1.20\) on page \(9\).
  Therefore, there exist a rational \(m\) in 
  \[
  	3 - \frac{1}{N} < m^2 < 3
  \]
  so \(m\in E\) but \(m\not\in \{U_n\}_{n\in I}\).
  We dont have a finite collection covering \(E\) so \(E\) is not compact.
  \par\smallskip
  Let \(x\in E\subset\mathbb{Q}\) where \(2 < x^2 < 3\).
\item
  Let \(E\) be the set of all \(x\in[0,1]\) whose decimal expansion contains
  only the digits \(4\) and \(7\).
  Is \(E\) countable?
  Is \(E\) dense in \([0, 1]\)?
  Is \(E\) compact?
  Is \(E\) perfect?
  \par\smallskip
  By cantor diagonalization argument, it can be shown that \(E\) is uncontable.
  Suppose \(E\) is countable and enumerate the elements of \(E\) as
  \(\{a_1,a_2,\cdots\}\) where \(a_i\) is a decimal with only fours and sevens.
  Let \(d\) be the decimal representation of of \(i\)-th digit from the
  \(a_i\)s.
  Then \(d\not\in E\) but \(d\in[0,1]\).
  Thus, \(E\) is uncountable.
  Suppose \(E\subset [0.4,0.8]\) is dense in \([0,1]\).
  Then every point of \([0,1]\) is a limit point of \(E\).
  However, \(E\) is closed and bounded so it contains all its limits points.
  Therefore, \([0,1]\setminus E\) can't be limit points of \(E\) so \(E\)
  is not dense in \([0,1]\).
  \(E\) is compact since it is closed and bounded.
  (missing \(E\) perfect proof)
\item
  Is there a nonempty perfect set in \(\mathbb{R}\) which contains no rational
  number?
\item
  \begin{exercise}[label = (\alph*), ref = \arabic{exercisei} (\alph*)]
  \item
    If \(A\) and \(B\) are disjoint closed sets in some metric space \(X\),
    prove that they are separated.
    \par\smallskip
    \(A\) and \(B\) being disjoint means that \(A\cap B = \varnothing\).
    A metric space \(E\) is closed if and only if \(E = \bar{E} = E'\cup E\).
    Therefore, \(A = \bar{A}\) and \(B = \bar{B}\).
    Then
    \begin{align*}
      A\cap B & = \bar{A}\cap B\\
              & = A\cap\bar{B}\\
              & = \varnothing
    \end{align*}
    Therefore, \(A\) and \(B\) are separated since
    \(\bar{A}\cap B = A\cap\bar{B} = \varnothing\).
  \item
    \label{2.19b}
    Prove the same for disjoint open sets.
    \par\smallskip
    Suppose on the contrary that \(A\cap\bar{B}\neq\varnothing\).
    Then there exists an \(x\in A\cap\bar{B}\).
    Let \(\epsilon > 0\) be given.
    Since \(x\) is an interior point of \(A\), there exists a neighborhood such
    that \(N_{\epsilon}(x)\subset A\).
    Since \(x\) is a limit point of \(B\), for all neighborhoods of \(x\), we
    have \(N_{\epsilon}(x)\cap B\neq\varnothing\).
    Therefore, \(A\cap B\neq\varnothing\) and we have reached a contradiction
    so disjoint open sets are separated.
  \item
    \label{2.19c}
    Fix \(p\in X\), \(\delta > 0\), define \(A\) to be the set of all
    \(q\in X\) for which \(d(p, q) < \delta\), define \(B\) similarly, with
    \(>\) in place of \(<\).
    Prove that \(A\) and \(B\) are separated.
    \par\smallskip
    Since \(A\) and \(B\) are disjoint open sets, by \cref{2.19b}, \(A\) and
    \(B\) are separated.
  \item
    Prove that every connected metric space with at least two points is
    uncountable.
    \textit{Hint: Use \cref{2.19c}.}
    \par\smallskip
    Let \(X\) be a connected metric space and let \(x,y\in X\).
    Let \(\epsilon > 0\) be given and set \(d(x,y) = \epsilon\).
    For all \(\delta\in (0, \epsilon)\), there exists a \(z\in X\) such that
    \(d(x,z) = \delta\).
    If this wasn't the case, metric space \(X\) would be made of two separated
    space similar to \cref{2.19c}.
    Since \((0,\epsilon)\) is an interval of real numbers, by Cantor's
    diagonalization argument, the interval is uncountable.
    Since there exists a \(z\in X\) for each \(\delta\), \(X\) is uncountable.
  \end{exercise}
\item
  Are closures and interiors of connected sets always connected?
  (Look at subsets of \(\mathbb{R}^2\).)
  \par\smallskip
  Let \(a > 0\).
  Let \(E = [-a,a]\times\{0\}\cup (-\infty,-a]\times\mathbb{R}\cup
  [a,\infty)\times\mathbb{R}\) be our connected set in \(\mathbb{R}^2\).
  The interior of \(E\) is
  \(\operatorname{Int}(E) = (-\infty,-a]\times\mathbb{R}\cup
  [a,\infty)\times\mathbb{R}\).
  Let \(A = (-\infty,-a]\times\mathbb{R}\) and
  \(B = [a,\infty)\times\mathbb{R}\).
  Then the interior of \(E\) is separated since
  \(A\cap\bar{B} = \bar{A}\cap B = \varnothing\).
  Therefore, the closures and interiors of connected sets are not always
  connected.
\item
  Let \(A\) and \(B\) be separated subsets of some \(\mathbb{R}^k\), suppose
  \(\mathbold{a}\in A\), \(\mathbold{b}\in B\), and define
  \[
  \mathbold{p}(t) = (1 - t)\mathbold{a} + t\mathbold{b}
  \]
  for \(t\in\mathbb{R}\).
  Put \(A_0 = \mathbold{p}^{-1}(A)\), \(B_0 = \mathbold{p}^{-1}(B)\).
  (Thus \(t\in A_0\) if and only if \(\mathbold{p}(t)\in A\).)
  \begin{exercise}[label = (\alph*)]
  \item
    Prove that \(A_0\) and \(B_0\) are separated subsets of \(\mathbb{R}\).
  \item
    Prove that there exists \(t_0\in (0, 1)\) such that
    \(\mathbold{p}(t_0)\not\in A\cup B\).
  \item
    Prove that every convex subset of \(\mathbold{R}^k\) is connected.
  \end{exercise}
\item
  \label{2.22}
  A metric space is \textit{separable} if it contains a countable dense subset.
  Show that \(\mathbb{R}^k\) is separable.
  \textit{Hint: Consider the set of points which have only rational
    coordinates}
\item
  \label{2.23}
  A collection \(\{V_{\alpha}\}\) of open subsets of \(X\) is said to be a
  \textit{base} for \(X\) if the following is true: For every \(x\in X\) and
  every open set \(G\subset X\) such that \(x\in G\), we have
  \(x\in V_{\alpha}\subset G\) for some \(\alpha\).
  In other words, every open set in \(X\) is the union of a subcollection of
  \(\{V_{\alpha}\}\).
  Prove that every separable metric space has a \textit{countable} base.
  \textit{Hint: Take all neighborhoods with rational radius and center in some
    countable dense subset of \(X\).}
\item
  \label{2.24}
  Let \(X\) be a metric space in which every infinite subset has a limit point.
  Prove that \(X\) is separable.
  \textit{Hint: Fix \(\delta > 0\), and pick \(x_1\in X\).}
  Having chosen \(x_1,\ldots,x_j\in X\), choose \(x_{j + 1}\in X\), if
  possible, so that \(d(x_i, x_{j + 1})\geq\delta\) for \(i = 1,\ldots,j\).
  Show that this process must stop after a finite number of steps, and that
  \(X\) can therefore be covered by finitely many neighborhoods of radius
  \(\delta\).
  Take \(\delta = 1/n\) \((n = 1,2,\ldots)\), and consider the centers of the
  corresponding neighborhoods.
\item
  Prove that every compact metric space \(K\) has a countable base, and that
  \(K\) is therefore separable.
  \textit{Hint: For every positive integer \(n\), there are finitely many
    neighborhoods of radius \(1/n\) whose union covers \(K\).}
\item
  Let \(X\) be a metric space in which every infinite subset has a limit point.
  Prove that \(X\) is compact.
  \textit{Hint: By \cref{2.23,2.24}, \(X\) has a countable base.}
  It follows that every open cover of \(X\) has a countable subcover
  \(\{G_n\}\), \(n = 1,2,\ldots\)
  If no finite subcollection of \(\{G_n\}\) covers \(X\), then the complement
  \(F_n\) of \(G_1\cup\cdots\cup G_n\) is nonempty for each \(n\), but
  \(\bigcap F_n\) is empty.
  If \(E\) is a set which contains a point from each \(F_n\), consider a limit
  point of \(E\), and obtain a contradiction.
\item
  \label{2.27}
  Define a point \(p\) in a metric space \(X\) to be
  \textit{condensation point} of a set \(E\subset X\) if every neighborhood of
  \(p\) contains uncountably many points of \(E\).
  Suppose \(E\subset\mathbb{R}^k\), \(E\) is uncountable, and let \(P\) be the
  set of all condensation points of \(E\).
  Prove that \(P\) is perfect and that at most countably many points of \(E\)
  are not in \(P\).
  In other words, show that \(P^c\cap E\) is at most countable.
  \textit{Hint: Let \(\{V_n\}\) be a countable base of \(\mathbb{R}^k\), let
    \(W\) be the union of those \(V_n\) for which \(E\cap V_n\) is at most
    countable, and show that \(P = W^c\).}
\item
  Prove that every closed set in a separable metric space is the union of a
  (possibly empty) perfect set and a set which is at most countable.
  \textit{(Corollary: Every countable close set in \(\mathbb{R}^k\) has
    isolated points.)
    Hint: Use \cref{2.27}}
\item
  \label{2.29}
  Prove that every open set in \(\mathbb{R}^1\) is the union of an at most
  countable collection of disjoints segments.
  \textit{Hint: Use \cref{2.22}}
  \par\smallskip
  Since \(\mathbb{R}^1\) contains a countable dense subset, \(\mathbb{R}^1\) is
  separable by \cref{2.22}
  This countable dense subset is \(\mathbb{Q}\).
  Let \(E\subset\mathbb{R}^1\) be open.
  Now \(E\cap\mathbb{Q}\) is a countable dense subset in \(E\).
  Pick open intervals in \(E\) around the the rationals in \(E\).
  Therefore, \(E\) is the union of these intervals which are countable.
  These intervals, unfortunately, overlap at the \(\inf\) and \(\sup\).
  Let \(x\in E\cap\mathbb{Q}\).
  Take the intervals in \(E\) that contain \(x\), \(I_1\).
  Let \(y\in E\cap\mathbb{Q}\).
  Again, take the intervals in \(E\) but construct the intervals as
  \(E\setminus I_1\) that contain \(y\), \(I_2\).
  Continuing on in this fashion, we obtain a countable collection of disjoint
  intervals \(\{I_1,I_2,\ldots\}\subset E\) and these intervals cover \(E\)
\item
  Imitate the proof of Theorem \(2.43\) to obtain the following results:
  \par\smallskip
  If \(\mathbb{R}^k = \bigcup_{n = 1}^{\infty}F_n\), where each \(F_n\) is a
  closed subset of \(\mathbb{R}^k\), then at least one \(F_n\) has a nonempty
  interior.
  \par\smallskip
  \textit{Equivalent statement:} If \(G_n\) is a dense open subset
  \(\mathbb{R}^k\), for \(n = 1,2,\ldots\) then
  \(\bigcap_{n = 1}^{\infty}G_n\) is not empty (in fact, it is dense in
  \(\mathbb{R}^k\)).
  \par\smallskip
  (This is a special case of Baire's theorem; see \cref{3.22}~\cref{ch3}, for
  the general case.)
  \par\smallskip
  Suppose on the contrary that \(\mathbb{R}^k = \bigcup_{n = 1}^{\infty}F_n\),
  where each \(F_n\) is a closed subset of \(\mathbb{R}^k\) and that each
  \(F_n\) has an empty interior.
  Let \(V_n = \bigcup_{i = 1}^nF_i\).
  For each \(n\), \(F_n\)  is closed so \(F_n^c\) is open.
  Then \(F_1^c\) is open.
  If \(F_1^c = \varnothing\), then \(F_1 = \mathbb{R}^k\) and
  \(F_1^{\circ}\neq\varnothing\); therefore, \(F_1^c\neq\varnothing\).
  Let \(K_1\) be a neighborhood of \(F_1^c\) such that
  \(\bar{K}_1\cap V_1 = \varnothing\).
  Similarly, let \(\bar{K}_n\cap V_n = \varnothing\).
  Let \(K_{n + 1}\) be the neighborhood in
  \(K_{n + 1}\setminus F_{n + 1}\neq\varnothing\) since
  \(F_{n + 1}^{\circ}\neq\varnothing\).
  Shrinking the neighborhood such that \(\bar{K}_{n + 1}\subset K_n\).
  Now \(\bar{K}_{n + 1}\cap V_{n + 1} = \varnothing\) so
  \(\bigcap_{n = 1}^{\infty}\bar{K}_n\) is disjoint for all \(F_n\).
  \par\smallskip
  Theorem \(2.39\): Let \(k\) be a positive integer.
  If \(\{I_n\}\) is a sequence of \(k\)-cells such that
  \(I_n\supset I_{n + 1}\) for \(n = 1,2,\ldots\), then
  \(\bigcap_1^{\infty}I_n\) is not empty.
  \par\smallskip
  Since \(\bar{K}_n\) is compact and \(K_n\supset K_{n + 1}\),
  \(\bigcap_1^{\infty}\bar{K}_n\neq\varnothing\) by theorem \(2.39\).
  \[
  x\in\bigcap_{n = 1}^{\infty}\bar{K}_n\subset
  \Bigl(\bigcup_{n = 1}^{\infty}F_n\Bigr)^c = \bigl(\mathbb{R}^k\bigr)^c
  = \varnothing
  \]
  and we have reached a contradiction since \(x\not\in\varnothing\).
  Thus, atleast one \(F_n\) has a nonempty interior.
\end{exercise}

%%% Local Variables:
%%% mode: latex
%%% TeX-master: t
%%% End:
